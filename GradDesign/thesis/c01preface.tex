\chapter{绪论}
\section{课题背景和意义}

基于物理的模拟是计算机动画、计算机辅助设计的基础。基于物理的模拟的本质是对于带有初边值条件的、大规模偏微分方程的求解。其中,偏微分方程,即控制方程,来源于牛顿第二定律与不同属性物体的本构方程结合。初值条件来源于用户设定的初始位置、速度等固定信息。边值条件主要来源于物理上物体之间的相互作用,以及用户在模拟过程中设置的交互。物理模拟中,动力学解算主要是指对于控制方程的解算,在不考虑边值条件的情况下,求解物体的的受力情况。边界条件解算主要是求解不同物体在交界面上的力学特征。

在图形学相关的物理模拟中,通常使用较为简单的本构模型,例如使用线性超弹性模型、不可压缩流体模型、弹簧质点模型来简化复杂的、真实的物理模型。因此,图形学的物理模拟中,动力学解算通常较为简单且易于实现。但由于场景、交互复杂,边值条件的解算成为在物理仿真中更复杂的部分。然而,相较于大规模科学计算软件的物理仿真,图形学应用中的用户交互更为丰富,边值条件的求解也更复杂,且图形应用具有强实时交互的要求,提升边值条件的解算效率,是物理仿真中的一大问题。

研究拥有不同材料控制方程物体之间的运动耦合方法,特别是流体-固体耦合方法是基于物理的模拟应用中的一个重要课题。流固耦合是指流体和固体之间相互作用的过程。在这个过程中,流体对固体施加压力,改变固体的形状和运动状态;同时,固体的形状和运动状态的变化也会影响流体的流动。基于惩罚函数法的流固耦合物理仿真,旨在通过流体、固体的控制方程对二者分别进行运动学建模,并利用惩罚函数对物体间相互作用进行统一建模,对复杂多物体场景的物理仿真提供了技术支撑。

研究流固耦合方法,主要目的在于优化现有计算机动画框架中的流体建模和求解算法的性能,提升计算机动画中流体的真实感,提升物理引擎仿真计算速度。而惩罚函数法是处理不同物体之间的约束的常见方法,其代码实现简单,在应用隐式时间积分器的时,拥有较高的数值稳定性。

传统的流固耦合方法常常基于上一时间步的固体速度和求解出的流体压强来计算下一时间步的物理量。尽管该方法可以将流固的计算分离,实现上较为简单,但也引入了较大的误差。例如,在重力场下的物体进入原本平衡的水面时,流体会晚于固体运动,从而发生流体到固体内部的穿透问题。基于惩罚函数法可以发现其中的穿透并将其转化为惩罚函数进行处理,能够有效消除穿透问题,并提供更为准确的流固耦合结果。

\section{本文研究内容}

本文主要提出了一个基于惩罚力实现流固运动耦合的算法。该算法采用物质点法实现流体模拟,有限元法和弹簧质点模型实现固体模拟。在耦合上,将物理上的碰撞视为约束,使用内点罚函数法模拟了流体粒子与固体的力相互作用。基于内点法的基本原理,本文应用了交替方向乘子法实现了计算加速。本文在多个场景下进行了测试,实验结果表明,该方法能够有效地解决流固之间的动力学耦合问题。

与此同时,本文还讨论了实验中使用的弹性体/布料/流体的运动学解算器(\ref{sec:dyn-solve}、\ref{sec:experiment-admm}节) 、二阶段碰撞检测的实现与测试(\ref{sec:collid-detect}节)。本文中采用的布料模拟算法、有限元算法和碰撞检测算法,相比先前的算法在相同条件下具有数倍的计算性能领先。

\section{论文各章节安排}

第\ref{chap:rel-works}章简要介绍了现有的流固运动学原理与求解方法、流固耦合的现有算法与惩罚函数法的应用。第\ref{chap:my-work}章详细描述了增量势能接触(IPC)及改进的ADMM-IPC流固耦合算法。第\ref{chap:experiment}章展示了ADMM-IPC算法的实现细节与测试结果。最后,第 \ref{chap:no-future} 章对于本文提出的方法进行了讨论与展望。
