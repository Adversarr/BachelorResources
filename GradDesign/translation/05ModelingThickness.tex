\section{厚度模型}

在其原始形式中,IPC 通过强制执行无符号距离的恒正来保持体积模型有无交的路径$d_k$,作为所有非相邻和非事件表面基元对 $k$ 之间的不变性。这适用于体积模型的接触情况,其中此约束允许存在任意接近但永不相交的表面。但是,对于同维模型,此约束已不再足够。当薄材料的3D变形减少到2D表面或1D曲线的变形时,曲面和曲线上的弹性可以很好地解析,但接触不能。忽略考虑同维接触中的有限厚度会产生不可接受的伪影(参见图2),并且显然无法捕获由薄结构相互作用形成的几何形状(参见图18和图8)。

我们首先观察到,将较大的$\hat d$(IPC接触力施加开始的阈值距离)应用于IPC中的同尺寸几何形状,可以模拟一个薄的响应层,该层在法向上抵抗压缩,从而形成弹性厚度。与这种弹性层一致,我们通常还需要一个核心厚度,超过这个厚度是不允许进一步压缩的。这是为了保证最小的有限(例如可见)厚度,即使在极端变形时也是如此(参见图2底部)。

为了模拟同维物体的接触厚度,我们构建了一个非弹性厚度模型,该模型将纯弹性层与硬质非弹性偏移到接触屏障相结合。对于非零偏移,$\xi$ ,几何形状保证彼此之间相隔 $\xi$ 。当距离低于$\xi + \hat d$ 时,弹性接触力施加,当距离在$\xi$处时发散。这里较大的$d$产生更大的弹性响应,而非零$\xi$保证最小厚度。即使在极端压缩下也是如此。

我们为每个表面元件 $i$ 配备有限厚度 $\xi_ i$ 。曲面上基元对 $k$ 的距离约束,在元素基元 $i$ 和 $j$ 之间形成,则为
\begin{equation}
  d_k(x) > \xi_k = \frac{\xi_i + \xi_j}{2}
\end{equation}
然后用圆角横截面近似同维材料的边界,而对于零厚度材料之间的相互作用,我们的距离约束减少到原始 IPC 约束的边界(例如,对于体积到体积接触)。最后,对于体积到共维接触,体积可以继续保持零厚度边界,同时与有限厚度同维边界相互作用。

对于厚度建模很重要的情况(参见图2,3-23),我们将ξk设置为材料的真实厚度或略小,然后在$\xi_k$附近设置$\hat d$以补偿剩余厚度,具体取决于每个应用的合理或所需的压缩量以及弹性响应。

\subsection{求解厚度边界}

为了实现数值稳健且高效的实现,IPC 应用距离平方(并通过适当的重新缩放)来计算等效的屏障函数。此更改将模型的派生接触屏障 $b(d_k(x), d)$ 换为使用 $b(d^2(x), d^2)$ 的等效重新缩放实现。在这里,我们通过平方距离直接应用加厚接触屏障如下
\begin{equation}
  b_\xi (d_k (x), \hat d) = b(d_k^2 (x) - \xi_k^2, 2\xi_k\hat d + \hat d^2),
\end{equation}
因此接触力在$d_k (x_ )= \xi_k$时正确发散,并且非零接触力只施加在比$\hat d + \xi_k$更近的中表面对之间。如果 $d^2 < \hat d^2$ ,则 $b(d^2 , \hat d^2 ) = −(d^2 − \hat d ^ 2 ) ^2 \ln(d^2 / \hat d^2)$,否则为 0。

在这里,势垒函数、梯度和黑森只需要用修改后的输入距离进行评估,而距离梯度和黑森保持不变,如下
\begin{equation}
  \frac{\mathbf d(d_k^2 - \xi_k^2)}{\mathbf dx} = \frac{\mathbf d d_k^2(x)}{\mathbf d x}
\end{equation}
用于计算摩擦力的接触力大小评估以及接触屏障的自适应刚度($\kappa$)更新也类似。

然后,大多数约束集计算需要简单且可比较的修改。对于空间散列,在定位体素之前,所有中表面基元的边界框在左下角和右上角的所有维度上按 $\xi_i/2$ 扩展。这使得 $\hat d$ 的哈希查询保持不变。接下来,对于宽相位接触对检测,查询距离 $\hat d$ 现在更新为 $\hat d + \xi_k$ 以检查边界框重叠。否则,等效地,边界框基元将像在空间哈希中一样扩展。最后,为了加速连续碰撞检测 (CCD) 查询,空间哈希构造也需要类似的边界框扩展,而对于其宽阶段,应用的间隙为 $\xi_k$(而不是 0)。

\subsection{对于CCD的挑战}

虽然上述厚度的初始修改很简单,但有限厚度和共维自由度给CCD查询带来了新的计算挑战。在这里,我们分析了这些挑战,并开发了一种新的CCD方法来解决这些问题。

标准形式的IPC在每次迭代中应用位置更新,以获得最小分离距离 s 和当前分离距离$d_k^{cur}$,在CCD评估时。这里的 $s \in(0, 1)$ 是一个保守的重缩放因子(通常设置为 s = 0.2 或 0.1),即使表面非常接近,也允许 CCD 查询避免相交。为了类似地求解有限厚度,我们的保守距离界限现在是$s(d_k^{cur} − \xi_k) + \xi_k$。

具体来说,障碍函数参数 $d_k(x) − \xi_k$ 必须始终保持正数。反过来,我们要求所有CCD评估,对于每个位移$p$,为我们提供足够准确的时间$t \in(0,1]$以满足正性。如果一对沿$p$发生撞击,我们需要一个时间t,以便缩放位移$tp$确保距离保持大于$\xi_k$,并尽可能接近目标分离距离$\xi_k + s(d_k^{cur} − \xi_k)$。

因此,使用有限厚度比不使用厚度更具挑战性。在接触 $d^{cur} − \xi_k$ 时,可以小至 $10^{−8 }m$(例如在大规模的压缩下),因此对于 $s\ge 10 ^{−9} m$ 左右的绝对误差是可以接受的。而在实践中,$\xi_k$通常处于$10^{−4} m$的尺度(例如,布料的厚度值为$\sim 3 \times 10^{−4}m$)。如果没有厚度 ($\xi_k = 0$),更新的距离以 $sd_k^{cur}$ 为目标(并且只需要严格大于 $0$)。换句话说,任何防止基元对交集的步长都是有效的,因此小于 100\% ≈ $10^{−9} m/s$ 的相对误差是可以接受的。另一方面,随着厚度的增加,更新的距离是针对$\xi_k + s(d_k^{cur} − \xi_k)$,因此,对于ξk的标准值,CCD评估的相对误差接近$0.01\%\approx 10^{ −9} m/(10 -4 m + s(10 −8 m))$,以避免相互穿透。

因此,对于现有方法来说,获得这种精度的CCD评估极具挑战性。作为起始示例,我们测试了Li等人的浮点CCD求解器,要求对两个具有挑战性的同维示例进行$s(d^{cur} − \xi_k)+ \xi_k$最小分离,厚度(见图10和21)$\xi_k>0$。在这里,即使我们完全删除保守的比例因子(即集合 $s = 0$),CCD 求解器也会错误地返回 $t = 0$ 影响时间 (TOI)(参见图 19)。注意,在执行此操作时,此错误会完全停止模拟进度。

\subsection{CCD 下界}

接下来,我们为CCD查询推导出一个有用的下限值,在数值上该下限值是鲁棒的,并且可以在浮点环境中有效地进行评估。这个下限提供了一个保守的、有保证的安全步长,也提供了一个明确的衡量标准来测试CCD查询的有效性:任何具有较小TOI的CCD类型评估显然都是失败的。在第 5.4 节中,我们应用此界限来推导出一个简单、有效且数值准确的显式 CCD 求解器,即使在薄材料仿真所需的具有挑战性的 CCD 评估中,该求解器也非常强大且高效。

在这里,在不损失一般性的情况下,我们将关注边对 $(x_0 ,x_1)$ 和 $(x_2,x_3)$ 之间的边-边情况,以及相应的位移 $p_0$、$p_1$、$p_2$ 和 $p_3$。任意点之间的距离函数分别由每条边上的$\gamma$和$\beta$参数化,则
\begin{equation}
  f(t, \gamma, \beta) = \| d(t, \gamma, \beta)\|
\end{equation}
然后,CCD评估寻求满足的最小正数实际值,满足以下条件:
\begin{equation}
  f_1(t) = \min_{\gamma,\beta} f(t, \gamma, \beta) = 0
\end{equation}
如果这样的$t$ 存在,我们称其为 $t_a$。参数 $(\gamma_a, \beta_ a) = \arg\min_{\gamma, \beta}f(t_a, \gamma, \beta)$依次给出在时间 $t_a$ 在两条边上碰撞的相应点。

我们可以直接精确地求解该问题。为了求解距离,已知$f_1(0)\le f(0, \gamma_a, \beta_a)$,那么三角不等式给出结果:
\begin{equation}
  t_a \ge \frac{f_1(0)}{\max(\|p_0\|, \|p_1\|) + 
  \max(\|p_2\|, \|p_3\|)}
\end{equation}
更广义的说,对于所有的查询,我们的下界$t_a$对应一个近似的距离函数,以及分母中两个基元中每个基元的最大位移之和。然后我们注意到,即使没有满足$f_1(t)= 0$的最小正时间(因此对区间没有影响),我们的边界仍然被明确定义为保守的步长。

接下来,我们观察到,也许令人惊讶的是,最先进的浮点CCD求解器可以并且将会返回小于我们下限的TOI结果,因此显然是错误的(见图19)。

最后,为了改善我们的边界,我们观察到它独立于参考系的选择而成立。因此,我们可以通过选择减少位移矢量$p_i$范数的帧来独立地进一步收紧每个 CCD 查询的界限。例如通过将每个$p_i$减去平均值 $\frac{1}{4}\sum_ip_i$。

\subsection{加性CCD}

使用我们计算的下限,我们现在应用 CA 策略为可变形体构建一种新的CCD算法,该算法具有有限偏移,该算法迭代更新并增加了我们在TOI的连续保守步骤中的下限。然后,所得的加性CCD(ACCD)方法可以鲁棒地求解有界TOI,单调接近每个CCD解,同时只需要显式调用来评估更新的基元位置之间的距离。

在每个 CCD 查询开始时,为了初始化 ACCD 算法(参见算法 1),我们首先将碰撞模板的位移居中在原点,以减少边界的分母,例如,$l_p = \max(\| p_0 \|, \| p_1 \|) + \max(\| p_2 \|,\|p_3 \|)$对于边缘-边缘对,因此增加我们可以安全采取的步长。如果没有相对运动($l_p = 0$),我们当然只是不返回碰撞,因此完整的单位步长是有效的。然后,我们根据当前的平方距离$d_{s q r}$和比例因子$s$计算$g = s(\sqrt{d_{s q r}} − \xi)$到偏移表面的最小间隔。为此,我们使用一个对取消错误更强大的公式。请参阅算法 1 中的第 8-9 行。
  
从最保守的冲击时间 $t = 0$(第 10 行)开始,我们创建节点位置的本地暂存器副本 $x_i$ ,并使用公式 10(第 11 行)初始化当前下限步长 $t_l$。

然后,我们进入迭代细化循环(第 12-21 行)以单调改进我们的 TOI 估计 t。在每次迭代中,我们使用当前步骤$t_l$(第13-14行)更新节点位置$x_i$的本地副本。如果这个新位置达到我们到偏移量的目标距离表面(变得小于g)我们已经收敛,前一个$t$是撞击时间,使距离刚好达到$g$(第17行)。如果没有,我们将当前 $t_l$ 添加到 $t$(第 18 行)来更新我们的 TOI 估计值。请注意,我们总是将第一个下限步长添加到 $t$(第 16 行),因为它保证不会使距离更接近 $g$ 。如果我们的TOI现在大于$t_c$,当前最小首次撞击时间(或者可以简单地设置为1),我们可以不返回碰撞(第19-20行)。否则,我们从更新的配置(第 21 行)计算一个新的局部下限 $t_l$(以 0.9 缩放以改善收敛性),并开始下一次迭代。

因此,ACCD提供了一个极其简单易实现、数值稳健的CCD评估。它只需要显式调用距离评估,因此没有数值上具有挑战性的寻根操作。反过来,ACCD能够支持厚度偏移和可控的精度,因此在CCD应用中灵活地调整性能与精度权衡。在第 6.6 节中,我们将 ACCD 与最先进的 CCD 求解器进行了比较,在那里我们看到它们都严重失败,导致交叉或微小的 TOI,从而在具有挑战性的示例中阻碍优化。反过来,在替代CCD方法成功的情况下,我们看到ACCD实现了类似且通常改进的计时性能。最后,我们指出,ACCD的停止标准要求$s > 0$(我们在所有示例中都应用$s = 0.1$)来提供有限终止。反过来,ACCD的目标不是计算精确的TOI,而是(如接触处理应用中实际要求的那样)获得可靠的,无交集的TOI步骤。

最坏情况的性能。如上所述,我们发现在实践中,特别是在我们所有具有挑战性的测试中,ACCD与稳定的性能(其他方法成功时相当或更快的时间,所有其他方法失败时的有效时间)收敛。然而,值得分析的是,应该有可能有无限的、最坏情况的表现。回想一下,我们选择参考系,使位移总和为 0。但是,如果一个基元有一个无法抵消的发散位移场,我们的分母l p可以保持很大。然后,如果基元的起始距离 $d$ 也很小,因此分子也相应很小,则在这种情况下 ACCD 的迭代计数可能会很大。虽然我们看到这种情况在实践中不会发生弹性动力学,但扩展ACCD以加速这些可能情况的收敛,从而获得有限的性能保证是一个有趣的未来步骤。




