\section{理论推导}

本节关注网格化、共维模型的组合解决方案,这些模型联合模拟并通过接触任意耦合。为此,我们将IPC模型推广到共维的自由度,因此必须解决薄模型带来的挑战,即这些模型既由接触和边界条件耦合,又通过接触和边界条件施加压力。在这里,我们首先介绍IPC方法在混合共维模型下的推广,然后在后续部分中构建方法的关键组件,以实现其仿真。

\textit{含接触弹性动力学}:对于弹性动力学,我们通过优化时间积分执行隐式时间步,通过线路搜索最小化增量电位(IP),确保稳定性和全局收敛。对于广义的的时间积分器,假设使用超弹性模型,从时间步长$n$更新到$n + 1$的IP定义为:
\begin{equation}
  E(x) = \frac{1}{2}\| x - \hat{x}^n \|_M^2 + 
  \alpha h^2\Psi (\beta x +\gamma x^n)
\end{equation}
时间步更新结果由$x^{n+1}=\mathrm{argmin}_xE(x)$给出。其中$h$是时间步长,$M$是质量矩阵,$\Psi$是弹性势能。比例系数$\alpha,\beta, \gamma \in [0, 1]$和预测的迭代初值$\hat x^n$决定了时间步使用的具体隐式积分格式。例如,我们主要是用图形学标准的隐式欧拉积分法,其决定于$\alpha,\beta = 1, \gamma = 0$且$\hat x^n = x^n + hv^n + h^2 M^{-1} f_{\mathrm{ext}}$。其他的隐式方法,例如隐式的中点法,与隐式欧拉法类似,可以由调整系数和初值来给出。

IPC给出了考虑了碰撞和耗散的势能函数,即IP:
\begin{equation}\label{eq2}
  E(x) = \frac{1}{2} \|x-\hat{x}^n\|_M^2 + \alpha h^2 \Psi(\beta x + \gamma x^n) + B(x) + D(x)
\end{equation}
其中的$B(x)$和$D(x)$分别对应了IPC模型中的碰撞障碍函数和摩擦势函数。碰撞障碍函数,$B$,使得所有图元对之间的距离在任何仿真步骤周恒大于零,而$D$提供相应的摩擦力。在 Li 等人之后,我们应用了一个自定义的牛顿型求解器,通过在每个牛顿迭代的每次行搜索中执行连续碰撞检测 (CCD) 过滤来最小化每个 IP,以确保整个过程中无交集轨迹。请参阅 Li 等人了解有关基本 IPC 算法和求解器实现的详细信息。

\textit{共维超弹性}:为了能够为耦合体积体、壳体、棒体和粒子提供统一的仿真框架,我们计算所有共维单元的质量和体积,方法是将它们视为相对于标准离散化的连续介质区域,从而将 IP 的总弹性能 $\Psi(x)$ 构建为
\begin{equation}\label{eq3}
  \Psi(x) =  \Psi_{vol}(x) + \Psi_{shell}(x) + \Psi_{rod}(x) 
\end{equation}
在这里,在不失一般性的情况下,作为代表性示例,我们对体积项($\Psi_{vol}$)应用固定共旋转弹性、离散壳铰链弯曲能量。结合各向同性或正交各向异性StVK,或壳的新胡克膜模型($\Psi_{shell}$),以及杆($\Psi_{rod}$)的离散杆拉伸和弯曲模型。我们选择这些模型作为比较和评估的最佳模型,给定现有代码中的模型标准。更一般地说,C-IPC框架与广泛的弹性和时间步进器选择无关,如第6节和Li等人所示。除了将所有能量与每个元素的精确体积权重正确积分外,我们还根据 Bergou 等人通过基尔霍夫杆理论进一步参数化杆弯曲模量,用于直接材料设置(有关详细信息,请参阅我们的补充文档)。将每个域的弹性相加到(\ref{eq3})中,我们的IPC模型现在通过摩擦接触直接耦合任意共维的对象(无需分裂)。具体来说,所有共维自由度现在都与它们各自的离散惯性和势能相关联,因此可以通过时间步进自由移动。反过来,它们又通过IPC类型的屏障耦合。C-IPC屏障现在包括来自所有表面(体积和壳)、棒和颗粒的所有点三角形和边缘-边缘对;来自所有杆节点和颗粒到杆段的点-边缘对;最后是所有粒子之间的点对。