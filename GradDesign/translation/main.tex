\documentclass[lang=cn]{ctexart}
\usepackage{cite}
\usepackage{hyperref}
\usepackage{amsmath}
\title{基于共维的增量势能碰撞模型}
\author{杨哲睿}

\begin{document}
\maketitle

\begin{abstract}
  我们扩展了增量势能模型,用于解决多个不同本征维度的弹性体间的碰撞处理问题。该方法为包含不同维度的几何体的带摩擦耦合问题提供了一个统一、无穿透的、鲁棒性高且稳定的模拟框架。扩展IPC模型到低维度几何题在处理应力计算、厚度建模和确定碰撞方面是具有挑战性的。面对这些挑战,我们在本文中提出了三个相关的方法。首先,我们引入了$C^2$连续的障碍模型,其直接将应力限制作为势能进行处理。该方法为布料模型提供了能量守恒的应力限制模型(各向同性和各向异性皆可),通过直接对于变形体和接触采用最小化增量势能的方法进行耦合。其次,为了捕获同维域的几何厚度,我们扩展了IPC模型以直接实施距离偏移。我们的处理严格保证壳体(或棒)的中表面(或棒表面)不会比应用的厚度值更近,即使这些厚度变得非常小。这使我们能够考虑同维结构接触行为中的厚度,从而可靠地捕获具有挑战性的接触几何形状。据我们所知,上述提到的模型中,其中一些以前从未模拟过。第三,同尺寸模型,尤其是具有建模厚度的模型,要求严格的精度要求,这对所有现有的连续碰撞检测(CCD)方法都提出了严峻的挑战。为了解决这些限制,我们开发了一种新的、有效的、易于实现的增量CCD(ACCD)方法,该方法应用保守的步长,反复细化形变基元的下限,收敛到撞击时间。这些贡献相结合,使共维IPC(C-IPC)成为可能。我们进行了广泛的基准测试,以验证我们的方法在捕获薄结构接触的复杂行为,和由此产生的体积效应方面的功效。在我们的实验中,C-IPC在所有测试样本中为所有时间步长获得了可行,收敛且无伪影的解决方案,从而产生了稳健的仿真。我们在极端变形、大时间步长和所有可能的共维结构域配对上极其紧密的接触下测试 C-IPC。最后,通过我们的应变极限模型,我们确认C-IPC保证在所有时间步长中所有合理(远低于 - 验证低至0.1\%)应变极限的非交集和应变极限满足。
\end{abstract}

\tableofcontents


\section{引言}

薄的材料,如布,头发和沙子无处不在。长期以来,对其进行动力学模拟仍然是计算机建模和动画中的一项关键任务。每个这样的固体通常最好通过相应的还原共维公式建模,分别由壳、棒和粒子组成。这样做可以在模拟自由度(DOF)更少的情况下提高效率。同样,它减轻了\textit{剪切锁定}带来的数值条件和严重的收敛挑战,否则如果模拟薄材料体积,我们会遇到这些挑战。但是,虽然我们可以通过同维模型很好地捕捉单个薄材料的弹性行为,但在实际尺寸网格上准确、一致地模拟它们的集体和接触行为,使用真实世界的材料和条件仍然具有挑战性。我们在此列举这些挑战。

首先,模拟应始终保持\textit{无相交}。即使允许同维几何体的轻微相交的时间步也可能会产生模拟无法恢复的问题。

其次,我们需要能量守恒、可控和准确的应变限制。虽然共维离散化减轻了剪切锁定,但它们仍然可能遭受严重的\textit{膜锁定}。例如,壳模型在布料的弯曲模式下继续遭受严重的数值刚度,除非中表面网格应用极高的(通常是无法实现的的尺寸)分辨率。。在这里,高阶FEM方法也可以帮助进一步减少锁定伪影,但它们增加了令人望而却步的计算成本。应变限制允许使用较软的布料材料,并结合严格的限制,以避免这些伪影,同时仍能恢复适当的布料响应。但是,如果应变限制没有准确且一致地应用,并且小心地与弹性动力学耦合,则通常会产生比修复更多的数值伪影。

第三,必须在接触中捕获几何厚度。虽然薄结构通常具有较小的相对厚度,其弹性行为可以通过共维自由度间接捕获,但正确和直接地模拟它们的接触厚度对于捕获准确的几何形状和体积行为至关重要。例如,考虑一副纸牌。单独考虑时,每张卡片的厚度几乎难以察觉。但是,堆叠时,它们的组合厚度大而清晰,不容忽视。虽然接触加工策略通常会引入厚度参数,但它们通常用作非物理故障保护,以减少碰撞处理的不准确性。正如Li等人所分析的那样,这些厚度无法通过统一的方法的被实现,而且必须根据场景和示例(例如,基于碰撞速度)进行更改,以避免模拟失败。

第四,所有共维(包括体积)都应在一个通用框架中无差别地,也没有特殊的处理地无缝模拟。精确的摩擦接触力应直接耦合所有几何类型,无论接触有多紧密或建模厚度有多薄。

第五,具有小但\textit{有限}厚度的共维模型之间的接触建模对所有碰撞检测算法提出了挑战。最关键的是,我们看到所有可用的CCD方法和代码都存在不可接受的错误或效率很低。无论是标准浮点寻根的CCD方法,还是精确CCD方法的最新发展,我们都看到了这一点。这些问题可归因于所提出的具有挑战性的潜在寻根问题众所周知的数值敏感性。在这里,最具体地说,我们看到IPC依赖于所有CCD计算正确,作为无交的保证。足够大的保守边界可以帮助解释这些不准确性,确保非交集,但它们这样做是以算法效率为代价的,因此在涉及共维自由度时甚至会完全停滞收敛。

本文提出了一种方法。据我们所知,该方法是第一个通过一致、可靠的摩擦接触动力学解决方案,来直接解决上述所有六个挑战的方法。我们的方法保证严格满足完全耦合的应变限制,无交集轨迹和可控的几何厚度分辨率(在薄材料尺度下),不受时间步长和接触条件严苛程度的影响。

\subsection{主要贡献}

具体来说,为了应对这些挑战,我们扩展了增量接触势能模型(IPC)用于含接触的弹性体动力学,以建模并求解由具有三个关键部分的任意共维自由度组合组成的系统。

\begin{itemize}
\item[\textbf{本构应变限制}]我们引入了一个$C^2$的罚函数模型,该模型直接将应变限制作为能量势,同时保持静止状态。这为布料提供了能量一致的应变极限模型(各向同性和各向异性),首次能够严格满足所有迭代和所有时间步长的应变极限不等式(验证低至 0.1\%),同时通过最小化增量电位完全直接耦合到弹性动力学和摩擦接触。因此,如第6.3节所示,我们避免了传统应变限制方法中由力分裂误差产生的伪影。
% TODO: 6.3节
  \item [IPC 厚度模型] 为了捕获接触的同维域的几何厚度,我们扩展了IPC模型以直接执行距离偏移。我们的处理严格保证壳(分别是棒和粒子)的表面(分别为中线和点)不会比应用的厚度值更近,即使这些厚度变得非常小(见第5节)。这使我们能够考虑同维结构接触行为中的厚度,从而捕获具有挑战性的接触效应。
  \item [增量CCD] 为了提供CCD所需的严格精度来解析具有厚度的共维模型,我们利用保守步场开发了一种新的,高效且易于实现的CCD(ACCD)方法。ACCD反复累加一个收敛于碰撞时间的下界,并且对于所需的极具挑战性的求解都是稳定的。虽然我们最直接关注的是ACCD在C-IPC框架内的应用,但正如我们在以下各节中所示,与之前的所有CCD算法(精确和浮点)相比,ACCD非常简单且易于实现,具有改进的性能,鲁棒性和保证,因此适合于所有使用CCD模块的应用中轻松更换。
\end{itemize}

这些共同构成了同维IPC(C-IPC)的核心。C-IPC 能够对所有共维进行统一仿真,包括弹性体积体、壳、棒和粒子,所有这些都通过精确求解的接触和摩擦耦合在一起。除了这些核心贡献外,我们还在许多新的和已存在的基准中测试、比较和分析C-IPC。我们确认C-IPC提供严格可控的几何厚度行为,并保证所有时间步的可行性。最后,C-IPC保持无交集,并严格满足所有计算轨迹上的所有设定应变限值,同样在我们广泛的基准测试中得到了验证。







\section{相关工作}

\subsection{壳、绳的模拟}

从Terzopoulos等人的开创性工作开始,共维模型的模拟,特别是壳和棒,一直是计算机图形学的重点。有效地模拟布料的复杂行为和头发现在是众多应用中的特别关键的任务。

用于模拟的计算流水线通常采用隐式或半隐式时间积分方法,应用了各种碰撞过滤器和惩罚来帮助解决接触处理问题。

为了提高性能,对 GPU 的支持和多分辨率网格法都在积极探索中。同时,为了提高模型的真度,数据驱动的材料估计,交替空间离散化,甚至是结合纱线和壳模拟的混合模型正被研究中。同样,布料模拟的可微性现在也成为神经网络训练相关应用的关键。

因此,一个关键的挑战是可靠地模拟壳并保证结果。Harmon等人引入了一种用于模拟壳的显式时间步长方法,该方法保证了所有时间步长的不相交。然后,这种方法需要较小的时间步长和简化的摩擦模型。在这里,我们提供了一种互补的、完全隐式的、可微的仿真方法,适用于所有共维模型,具有精确的摩擦接触,保证非相交,以及非反转和(课选的)应变极限满足,可以在所有合理的时间步长下进行。

\subsection{应变限制}

应变限制方法试图对膜变形施加限制。长期以来,人们一直提出了广泛的模型和算法来提供应变限制。为了理解最近方法的行为和局限性,我们按以下方式对它们进行分类:1)约束选择(等式约束或不等式约束);2)受约束的DOF类型(这些通常是基于边缘或奇异值);3)算子拆分模型;4) 支持施加强制约束的求解器。

通过这种细分,我们看到对基于长度的措施的平等约束仍然是一个一致的约束选择。Goldenthal等人对四边形网格边缘长度应用双边约束,并提出了一种快速投影方法来校正预测的位移。English和Bridson等人采用不一致性策略,对三角形边缘中点距离应用相等约束,并扩展快速投影以支持 BDF-2 积分器。Thomaszewski 等人在三角形上应用滞后、旋转同步的小应变来定义等式约束,并通过G-S或Jacobi迭代的后投影法来强制施加应变限制。Chen 和 Tang 同样定义了三角形边长度的等式约束。

通常,可以通过将变形梯度(例如每个三角形的形变梯度)的奇异值约束到固定且有限的范围内,来获得对基于边缘的约束的改进。然而,无论细节如何,相等约束始终保持活动状态,因此简单的自由度计数通常解释了在实施这些应变约束时经常遇到的膜锁定伪影。在这里,替代约束自由度选择可以通过降低约束与自由度的比率来帮助缓解这个问题,但也可以引入新的挑战,例如通过不合格的网格。作为双边的不平等约束的替代方案,以上限和下限的形式早已得到应用,而最近,该约束也被转化为仅考虑上限的情况。但是无论边界如何设置,单边约束只有在其应变措施达到极限时才会激活,因此过度约束系统的可能性降低。

虽然方法明显不同,但我们观察到,除了Chen和Tang的无摩擦弹性静力学的最小二乘近似之外,所有方法目前都采用时间步长分割。在这里,我们看到,在应用约束投影以同时求解接触约束和应变限制之前,通常应用两步法首先求解弹性动力学。或者,引入更多误差,我们还看到三步分裂,其中弹性动力学,接触和应变极限的顺序求解都是解耦的,因此每个时间步独立解决[Narain等人,2012]。因此,对于后续的三步法,其不能同时满足应变极限和接触约束,而对于前两步法,通过分开求解弹性和约束引入的误差也会产生不可接受的伪影;请参阅第 6 节。

最后,我们注意到,据我们所知,没有任何方法(无论约束类型或分裂选择如何)使用约束求解器来保证应变极限的执行。由于约束执行错误在模拟场景甚至单个时间步长中都有所不同,这会导致每个场景和步骤的材料行为发生不可控和不一致的变化。为了实现一致、有效的应变限制C-IPC引入了一个完全耦合的、基于不等式的模型,并严格保证应变极限满足。

\subsection{厚度建模}

通过同维几何建模的薄体的厚度通常比其他尺寸小几个数量级。因此,直接忽略接触厚度是很有吸引力的(也是常见的)。然而,为了正确捕获薄材料的相互作用,我们必须考虑接触厚度的几何效应。我们到处都能看到这种情况,例如当扑克牌堆成一堆(图8)或面条装满碗时(图10)。

捕获厚度的直接策略是硬壳法,该方法使用全平移DOF对薄材料进行体积建模。然而,在这样做的过程中,线性有限元会受到剪切锁定伪影的影响。虽然高阶有限元可以缓解这些收敛问题,但剪切锁定伪影仍然不容易完全避免,而计算成本显着增加。为了应对这些挑战,该方法通常采用假设的自然应变方法的减少集成,以减轻线性单元的剪切锁定。但是,这种增加的复杂性通常不会与共维建模竞争。

为处理共维模型的接触,接触处理策略通常会引入类似厚度的参数来偏移约束,从而减少碰撞处理的不准确性;例如,参见Narain等人和Li 等人以获取最近的例子。然而,正如Li等人所分析的那样,这些厚度无法统一地进行强制施加,而且必须根据场景和数据(例如基于碰撞速度)进行启发式更改,以避免交叉点和不稳定性导致的模拟失败。反过来,虽然它们有时有助于提高接触分辨率,但这些调整后的参数不能可靠地应用于模型一致和不断变化的厚度行为。请参阅第 6.5 节。

我们扩展了IPC模型,以捕获具有偏移屏障的同维域的几何厚度,从而保证与中表面以及中线和点几何形状的最小间距。反过来,这使得在薄接触材料之间的相互作用中对厚度行为进行可靠和一致的建模。

\subsection{连续碰撞检测}

连续碰撞检测(CCD)方法长期以来一直被用作检查相交网格边界的方法。Provot 首先通过三次方程的解来测试共面性,然后执行重叠检查以检测碰撞,从而首次发现线性位移点三角形和边边对的冲击。

到目前为止,浮点CCD的标准公式已经广泛应用并随着技术进步进行了微调。然而,虽然使用浮点运算进行寻根可以提高效率,但重大的数值误差仍然会产生不可接受的错误结果。当距离较小或维数退化时。在这里,虽然从浮点切换到有理数有助于避免舍入问题,但如果不特别小心,成本可能会增加得令人无法接受。

最近的方法解决了CCD的准确性通过应用精确算法来提高鲁棒性与效率。或者,其他人使用要求的保守间隔距离进行保守CCD,以更好地避免相互渗透。后一种策略用于体积元的IPC,使用稳定的浮点CCD实现作为基础求解器。

在这里,为了在C-IPC中建模厚度,我们需要CCD求解,该求解可以保持表面、中线和点元素之间的有限间隔距离。反过来,我们依靠精确地保留偏移表面之间数量级的较小距离来计算精确的接触力。这要求CCD算法有更高的的准确性和稳健性。我们看到所有可用的现有CCD方法和代码中都出现了不可接受的错误,导致故障。我们在标准浮点寻根CCD方法以及最近的精确CCD方法中都看到了这一点。反过来,我们看到这些故障会产生交叉点,减慢收敛速度,并且经常完全停止仿真进度。请参阅第 6.6 节。

作为寻根的替代方案,保守前进(CA)方法已被提出。该算法迭代推进刚体和/或铰接体,直到它们比预定义的小距离更近。从 Mirtich 关于凸刚体的工作 [1996] 开始,这是通过反复计算冲击时间下限(TOI)然后对边界采取保守步骤来实现的。

为了对可变形体轨迹进行鲁棒CCD评估,我们得出了具有任意位移的变形网格基元对的冲击时间下限,并将其应用于CA框架中,以开发一种新的,易于计算的,数值鲁棒的浮点加性CCD(ACCD)算法。在 CA 框架下,ACCD 单调地接近影响时间,没有容易出错的直接寻根。在第6节中,我们确认ACCD在所有其他方法都失败的各种具有挑战性的案例中高效准确地成功,并在浮点CCD方法可以成功的情况下找到类似且通常更好的性能。最后,我们验证了ACCD也适用于C-IPC框架之外的CCD模块,具有更高的效率和鲁棒性。

\subsection{共维建模}

在通用框架中统一仿真所有共维物体,对于仿真效率和准确性至关重要。Martin 等人 专注于统一的弹性模型。它们推导出Elaston,一种高阶积分规则,用于测量沿所有轴的拉伸、剪切、弯曲和扭曲,而不区分余维。Elaston可以精确捕获各种弹塑性行为,而接触力则由逐点惩罚确定,物体由一组球体表示。Chang 等人通过相等约束定义不同共维域之间的所有连接,从而解决混合维弹性体的统一问题。在这里,Bridson等人的碰撞处理算法随后应用于解决接触。物质点法(MPM)还提供了所有协域及其之间接触的通用建模。MPM 离散拉格朗日粒子上的弹性,同时求解欧拉网格自由度上的动量平衡。然后,同维物体之间的接触通过欧拉网格直接解析为速度流。但是,接触中的粘连和间隙误差是MPM中众所周知的伪影,如果网格分辨率太低,则可能是不可接受的。Han等人用拉格朗日自由度对这些伪影进行了研究和缓解。

基于位置的动力学 (PBD)还可以实现具有不同同尺寸的实体的无缝、统一耦合。在这里,本构模型和接触都被解析为约束,随着约束复杂性的增加,这些约束被迭代处理以实现有效的时间积分,但代价是其无法对于精度有准确控制。PBD与XPBD的扩展,即投影动力学,以及通过ADMM的进一步推广都同样提供了协同模拟多种协域的平台。最近的增强功能现在可以改善摩擦力并提高效率。然而,这些方法都缺乏对于迭代收敛的保证,并且采用固定的迭代上限,因此精度和鲁棒性(弹性和接触的分辨率)对计算效率的基本权衡仍然存在。在实践中,这意味着可以而且将会遇到数值不稳定和数值爆炸,特别是在具有挑战性的场景中,例如具有较大的时间步长、刚度、变形或速度,如 Li 等人所示。相反,C-IPC以收敛性和稳定性为目标,能够同时直接模拟所有共维。耦合是通过精确、无相交的接触在任何和所有共维配对之间的相互作用提供的。





\section{理论推导}

本节关注网格化、共维模型的组合解决方案,这些模型联合模拟并通过接触任意耦合。为此,我们将IPC模型推广到共维的自由度,因此必须解决薄模型带来的挑战,即这些模型既由接触和边界条件耦合,又通过接触和边界条件施加压力。在这里,我们首先介绍IPC方法在混合共维模型下的推广,然后在后续部分中构建方法的关键组件,以实现其仿真。

\textit{含接触弹性动力学}:对于弹性动力学,我们通过优化时间积分执行隐式时间步,通过线路搜索最小化增量电位(IP),确保稳定性和全局收敛。对于广义的的时间积分器,假设使用超弹性模型,从时间步长$n$更新到$n + 1$的IP定义为:
\begin{equation}
  E(x) = \frac{1}{2}\| x - \hat{x}^n \|_M^2 + 
  \alpha h^2\Psi (\beta x +\gamma x^n)
\end{equation}
时间步更新结果由$x^{n+1}=\mathrm{argmin}_xE(x)$给出。其中$h$是时间步长,$M$是质量矩阵,$\Psi$是弹性势能。比例系数$\alpha,\beta, \gamma \in [0, 1]$和预测的迭代初值$\hat x^n$决定了时间步使用的具体隐式积分格式。例如,我们主要是用图形学标准的隐式欧拉积分法,其决定于$\alpha,\beta = 1, \gamma = 0$且$\hat x^n = x^n + hv^n + h^2 M^{-1} f_{\mathrm{ext}}$。其他的隐式方法,例如隐式的中点法,与隐式欧拉法类似,可以由调整系数和初值来给出。

IPC给出了考虑了碰撞和耗散的势能函数,即IP:
\begin{equation}\label{eq2}
  E(x) = \frac{1}{2} \|x-\hat{x}^n\|_M^2 + \alpha h^2 \Psi(\beta x + \gamma x^n) + B(x) + D(x)
\end{equation}
其中的$B(x)$和$D(x)$分别对应了IPC模型中的碰撞障碍函数和摩擦势函数。碰撞障碍函数,$B$,使得所有图元对之间的距离在任何仿真步骤周恒大于零,而$D$提供相应的摩擦力。在 Li 等人之后,我们应用了一个自定义的牛顿型求解器,通过在每个牛顿迭代的每次行搜索中执行连续碰撞检测 (CCD) 过滤来最小化每个 IP,以确保整个过程中无交集轨迹。请参阅 Li 等人了解有关基本 IPC 算法和求解器实现的详细信息。

\textit{共维超弹性}:为了能够为耦合体积体、壳体、棒体和粒子提供统一的仿真框架,我们计算所有共维单元的质量和体积,方法是将它们视为相对于标准离散化的连续介质区域,从而将 IP 的总弹性能 $\Psi(x)$ 构建为
\begin{equation}\label{eq3}
  \Psi(x) =  \Psi_{vol}(x) + \Psi_{shell}(x) + \Psi_{rod}(x) 
\end{equation}
在这里,在不失一般性的情况下,作为代表性示例,我们对体积项($\Psi_{vol}$)应用固定共旋转弹性、离散壳铰链弯曲能量。结合各向同性或正交各向异性StVK,或壳的新胡克膜模型($\Psi_{shell}$),以及杆($\Psi_{rod}$)的离散杆拉伸和弯曲模型。我们选择这些模型作为比较和评估的最佳模型,给定现有代码中的模型标准。更一般地说,C-IPC框架与广泛的弹性和时间步进器选择无关,如第6节和Li等人所示。除了将所有能量与每个元素的精确体积权重正确积分外,我们还根据 Bergou 等人通过基尔霍夫杆理论进一步参数化杆弯曲模量,用于直接材料设置(有关详细信息,请参阅我们的补充文档)。将每个域的弹性相加到(\ref{eq3})中,我们的IPC模型现在通过摩擦接触直接耦合任意共维的对象(无需分裂)。具体来说,所有共维自由度现在都与它们各自的离散惯性和势能相关联,因此可以通过时间步进自由移动。反过来,它们又通过IPC类型的屏障耦合。C-IPC屏障现在包括来自所有表面(体积和壳)、棒和颗粒的所有点三角形和边缘-边缘对;来自所有杆节点和颗粒到杆段的点-边缘对;最后是所有粒子之间的点对。
\section{本构应变限制}

在这里,我们首先构建一个新的本构障碍模型,该模型直接执行应变限制,同时保持静止状态的一致性。我们从一种配方开始,该配方增加了现有的膜能量,并在一般各向同性的情况下增加了应变限制电位(第4.1节)。然后,我们在 4.2 节中演示了它的扩展,以直接增强具有应变极限的正交各向异性 STVK 膜。

\subsection{各向同性本构应变限制}

我们将每个元素 $t$ 的各向同性应变限制约束定义为:
\begin{equation}
  \sigma _i ^t < s, \quad \forall i
\end{equation}
每个三角形 t 的变形梯度的奇异值分解为$F^t = U^t\Sigma^t{V^t}^T$,其中$\Sigma^t = \mathrm{diag} (\sigma_1^t, \sigma_2^t)$。约束界$s$表示指定的应变上限,其实际通常选择用于具有 $s\in [1.01, 1.1]$ 的布料。

应变限制需要局部支集。它只应在拉伸过程中当应变接近施加的上限时施加约束力。否则,它应该保持底层膜模型不变。这类似于只有在几乎要接触的情况下才施加的接触力。因此,我们从IPC的接触势障碍开始,构建一个类似的$C^2$平滑、且紧的障碍函数,用于应变限制。对于每个障碍$\sigma_i^t$定义为:
\begin{equation}
  b_i^{st}(x) = \begin{cases}
    -(\frac{\hat s - \sigma_i^t}{s - \hat s})^2\ln(\frac{\hat s - \sigma_i^t}{s - \hat s})& \sigma_i^t > \hat s\\
    0 & \sigma_i^t \le \hat s
  \end{cases}
\end{equation}
因此,仅当应变超过施加的一个小的应变阈值 $s$ 时才会激活。例如,参见势垒能量的插图,应变极限 $s = 1.1$ 和阈值 $\hat s = 1$。

现在,有了障碍函数,我们接下来考虑将它们直接视为约束(如原始障碍和内部点方法)。但是,这样做会简单地将这些障碍相加到元素上,因此在我们更改网格时会获得不一致的行为。相反,为了提供一致的行为,我们以能量密度的形式施加了应变限制,以积分在布料体积上以获得势能函数:
\begin{equation}
  \Psi _{\mathrm{SL}} (x) = \sum_i \kappa_s \int _ \Omega b_i^{st} (x) dV
  \approx \kappa_s \sum_{t, i} V^t b_i^{st}(x)
\end{equation}
其中$\kappa_s$是障碍函数的刚度。对于我们的近似,我们使用$V^t = A^t \xi^t$,其中$A^t$和$\xi^t$分别是三角形t的面积和厚度。我们的最终各向同性的应变限制势能$\Psi_{\mathrm {SL}}$在局部支持下为$C^2$,可以简单地添加到公式\ref{eq3}中的总电位中。反过来,通过优化公式\ref{eq2}中的IPC,在每个时间步直接处理应变限制,同时确保在每个行搜索步骤中不违反应变限制约束。在大多数三角形保持在应变极限阈值以下的步骤中,几乎不需要额外的计算。同时,当拉伸增加时,来自新激活的势垒的必要非零项被添加到系统中,因此,正如我们在第 6 节中所示,在平衡所有施加力的同时严格执行应变限制。

\textit{应变极限刚度}:定义屏障电位后,我们现在指定设置其刚度。对于我们的应变极限模型,我们应用的软化夹紧屏障,其灵感来自IPC对接触势垒能量的平滑。然而,在这里,请注意应变限制的额外成分是应用的 $1/(s − \hat s)^2$ 因子。这种缩放使我们能够将我们的屏障直接应用于极限归一化应变,计算的公式为 $y = (\hat s − \sigma_i^t )/(s −\hat s)$
\begin{equation}
  \hat b_i ^{st}(y) = \begin{cases}
    - y ^ 2\ln(1 + y) & y < 0,\\
    0 & y\ge 0.
  \end{cases}
\end{equation}
反过来,由于我们施加的应变极限($s$)和/或截断阈值($\hat s$)随应用而变化,因此势垒函数,即$y$保持不变。只有从 $\sigma_i^t$ 到 $y$ 的线性映射的梯度变化。这使我们能够在所有不同的应变极限和夹紧阈值的选择上应用单一、一致的初始势垒刚度$\kappa_s$(我们为所有示例使用 $1KPa$)。在这里,障碍势能函数曲线每次都简单地线性重新调整到不同的应变范围,从而为系统提供一致的调节。最后,为了避免当前的应变和需要施加的应变极限之间的微小间隙带来的数值问题(例如,当施加极端边界条件如图9和图1所示时),我们在需要时按照IPC的刚度调整策略调整势垒刚度。从我们的初始$κ^s$开始,每当三角形$t$的应变间隙$s − \sigma_i^t$在两次连续迭代中小于$10^{-4}(s − \hat s)$时,我们将其增加$2\times$(最大界限的$\kappa_s = 0.1MPa$)。请注意,应用此调整时会改变应变限制能量。但是,在极端情况下,这种情况仅发生在实际限制附近。在这里,应变极限力施加约束,而这种适应性提供了改进的数值条件。一旦离开极限,屏障就会恢复到一致的能量,在阈值$s$以下没有任何影响。

\subsection{各向异性本构应变限制}

上面我们已经构建了应变限制作为各向同性本构模型。我们形成了一个可以积分的阻挡能,因此直接添加到势能中,以增强具有硬应变限制的现有膜模型。完成这项工作的关键是应用我们的 $C^2$ 连续函数,以便应变限制的应用不会改变稳定状态下的梯度或Hessian。

或者,我们可以应用类似的本构应变限制策略直接修改膜弹性模型以包括应变极限。为此,我们只需将各向异性膜模型替换为可防止违反应变极限的势垒能,同时匹配原始膜能量梯度和静止时刻的Hessian。

第二种策略的动机是需要将应变限制纳入可用的数据驱动模型中,这些模型已被构建为仔细拟合测量的布料数据。在这里,我们展示了克莱德等人的各向异性且是数据驱动模型的具体应用。其本构模型的能量函数为
\begin{equation}
  \tilde\psi(\tilde E_{11},\tilde E_{12},\tilde E_{22}) = \frac{a_{11}}{2} \eta_1(\tilde E_{11}^2)+\frac{a_{22}}{2}\eta_3(\tilde E_{22}^2)
  + a_{12}\eta_2(\tilde E_{11}\tilde E_{22})+G_{12}\eta_4(\tilde E_{12}^2)
\end{equation}
其中 $\hat E = D^TED$ 是简化并对齐的格林-拉格朗日应变,$D$的列向量是壳在材料空间中的切基和法向基。函数$η$是多项式函数的总和,其实值指数$α _{j i}$ 满足$\eta(0) = 0$ 和 $\eta'(0) = 1$。在这里,第一个约束强制执行零能量、零应力稳定配置,而第二个约束允许参数 $a_{\alpha\beta}$ 和 $G _{12}$ 与无穷小应变下的线性弹性之间的自然对应关系。克莱德等人使用
\begin{equation}
  \eta_j(x) = \sum_{i = 1}^{d_j} \frac{\mu_{ji}}{\alpha_{ji}}((x+1)^{\alpha_{ji}} - 1).
\end{equation}
然后,他们的测量数据仅限于布料断裂极限或弹性范围内的变形。超出此范围,则不太可能进行有意义的外推,因为可能在范围内过度拟合(拟合的指数$α_{ji}$ 的范围可达 $10^5$ )。为了解决这一限制,Clyde等人提出了一种二次外推法进行仿真。然而,这种推断在物理上是没有意义的。为了有效地应用数据驱动的建模,要么应该捕获超出此范围的断裂,要么施加稳定和可控的应变极限以保持在边界内。在这里,我们专注于在尊重基础模型拟合的同时,以可控方式施加应变极限。

\textit{障碍函数定义}。从相同的本构模型开始,我们用障碍函数重新定义$\eta$为
\begin{equation}
  \eta(E) = -E^{\max} \log ((E^{\max} - E) / E^{\max}).
\end{equation}
为了一致性。请注意,此函数需要再次满足$\eta(0) = 0$ 和 $\eta' (0)= 1$,因此对弹性有效。该势垒在 $E^{\max}$ 时,$\eta$也会发散,从而确保 $E$ 永远不会超过测量的应变极限界限。因此,它可以捕获数据驱动的应变极限,同时避免外推。在这里,由于基础模型是各向异性的,因此在各个剪切方向都可以施加不同的测量 $E$ 最大边界,从而能够保留所有应变极限的测量各向异性。

然后,自由模型参数$a_{\alpha\beta}$和$G_{12}$在$\nabla^2\psi(0,0,0)$处与基础模型的拟合匹配。有关详细信息,请参阅补充文档。请注意,这个公式使我们不必计算具有大实指数的敏感(且可能昂贵的)多项式。有关所提出的模型的实验,请参见第 6.2 节。最后,我们注意到,与我们在上一节的各向同性案例中演示的策略(我们直接研究了变形梯度的上限)相反,这里的第二能量对于拉伸和压缩是对称的。

\subsection{用于应变限制的线搜索方法}

如上所述,基于CCD的限制器对于确保线路搜索与我们的接触屏障一致至关重要。现在我们添加了应变限制屏障,我们必须进一步确保每个线搜索也不会违反施加的应变限制。对于各向同性极限,尽管 $2 \times 2$ 个 SVD 具有闭式解,但它不能为可行的步长计算提供多项式方程。因此,在C-IPC线搜索中,在我们计算出无交集起始步长后,如果我们在回溯过程中检测到应变限制违规,我们只需将步长减半。我们重复,直到获得满足能量降低和应变极限的步长。在实践中,因为我们的牛顿步长包括来自Hessian和梯度的应变限制的高阶信息,我们观察到IPC搜索方向在自然避免应变限制方面是有效的。在所有情况下,如果需要,到目前为止,我们观察到最多需要三个回溯步骤就可以限制应变。对于我们的各向异性模型,我们目前也应用相同的回溯策略。然而,虽然目前不需要考虑效率,但我们确实注意到,在这里我们可以制定二次方程,适合直接计算满足步长的最大可行应变极限。















\section{厚度模型}

在其原始形式中,IPC 通过强制执行无符号距离的恒正来保持体积模型有无交的路径$d_k$,作为所有非相邻和非事件表面基元对 $k$ 之间的不变性。这适用于体积模型的接触情况,其中此约束允许存在任意接近但永不相交的表面。但是,对于同维模型,此约束已不再足够。当薄材料的3D变形减少到2D表面或1D曲线的变形时,曲面和曲线上的弹性可以很好地解析,但接触不能。忽略考虑同维接触中的有限厚度会产生不可接受的伪影(参见图2),并且显然无法捕获由薄结构相互作用形成的几何形状(参见图18和图8)。

我们首先观察到,将较大的$\hat d$(IPC接触力施加开始的阈值距离)应用于IPC中的同尺寸几何形状,可以模拟一个薄的响应层,该层在法向上抵抗压缩,从而形成弹性厚度。与这种弹性层一致,我们通常还需要一个核心厚度,超过这个厚度是不允许进一步压缩的。这是为了保证最小的有限(例如可见)厚度,即使在极端变形时也是如此(参见图2底部)。

为了模拟同维物体的接触厚度,我们构建了一个非弹性厚度模型,该模型将纯弹性层与硬质非弹性偏移到接触屏障相结合。对于非零偏移,$\xi$ ,几何形状保证彼此之间相隔 $\xi$ 。当距离低于$\xi + \hat d$ 时,弹性接触力施加,当距离在$\xi$处时发散。这里较大的$d$产生更大的弹性响应,而非零$\xi$保证最小厚度。即使在极端压缩下也是如此。

我们为每个表面元件 $i$ 配备有限厚度 $\xi_ i$ 。曲面上基元对 $k$ 的距离约束,在元素基元 $i$ 和 $j$ 之间形成,则为
\begin{equation}
  d_k(x) > \xi_k = \frac{\xi_i + \xi_j}{2}
\end{equation}
然后用圆角横截面近似同维材料的边界,而对于零厚度材料之间的相互作用,我们的距离约束减少到原始 IPC 约束的边界(例如,对于体积到体积接触)。最后,对于体积到共维接触,体积可以继续保持零厚度边界,同时与有限厚度同维边界相互作用。

对于厚度建模很重要的情况(参见图2,3-23),我们将ξk设置为材料的真实厚度或略小,然后在$\xi_k$附近设置$\hat d$以补偿剩余厚度,具体取决于每个应用的合理或所需的压缩量以及弹性响应。

\subsection{求解厚度边界}

为了实现数值稳健且高效的实现,IPC 应用距离平方(并通过适当的重新缩放)来计算等效的屏障函数。此更改将模型的派生接触屏障 $b(d_k(x), d)$ 换为使用 $b(d^2(x), d^2)$ 的等效重新缩放实现。在这里,我们通过平方距离直接应用加厚接触屏障如下
\begin{equation}
  b_\xi (d_k (x), \hat d) = b(d_k^2 (x) - \xi_k^2, 2\xi_k\hat d + \hat d^2),
\end{equation}
因此接触力在$d_k (x_ )= \xi_k$时正确发散,并且非零接触力只施加在比$\hat d + \xi_k$更近的中表面对之间。如果 $d^2 < \hat d^2$ ,则 $b(d^2 , \hat d^2 ) = −(d^2 − \hat d ^ 2 ) ^2 \ln(d^2 / \hat d^2)$,否则为 0。

在这里,势垒函数、梯度和黑森只需要用修改后的输入距离进行评估,而距离梯度和黑森保持不变,如下
\begin{equation}
  \frac{\mathbf d(d_k^2 - \xi_k^2)}{\mathbf dx} = \frac{\mathbf d d_k^2(x)}{\mathbf d x}
\end{equation}
用于计算摩擦力的接触力大小评估以及接触屏障的自适应刚度($\kappa$)更新也类似。

然后,大多数约束集计算需要简单且可比较的修改。对于空间散列,在定位体素之前,所有中表面基元的边界框在左下角和右上角的所有维度上按 $\xi_i/2$ 扩展。这使得 $\hat d$ 的哈希查询保持不变。接下来,对于宽相位接触对检测,查询距离 $\hat d$ 现在更新为 $\hat d + \xi_k$ 以检查边界框重叠。否则,等效地,边界框基元将像在空间哈希中一样扩展。最后,为了加速连续碰撞检测 (CCD) 查询,空间哈希构造也需要类似的边界框扩展,而对于其宽阶段,应用的间隙为 $\xi_k$(而不是 0)。

\subsection{对于CCD的挑战}

虽然上述厚度的初始修改很简单,但有限厚度和共维自由度给CCD查询带来了新的计算挑战。在这里,我们分析了这些挑战,并开发了一种新的CCD方法来解决这些问题。

标准形式的IPC在每次迭代中应用位置更新,以获得最小分离距离 s 和当前分离距离$d_k^{cur}$,在CCD评估时。这里的 $s \in(0, 1)$ 是一个保守的重缩放因子(通常设置为 s = 0.2 或 0.1),即使表面非常接近,也允许 CCD 查询避免相交。为了类似地求解有限厚度,我们的保守距离界限现在是$s(d_k^{cur} − \xi_k) + \xi_k$。

具体来说,障碍函数参数 $d_k(x) − \xi_k$ 必须始终保持正数。反过来,我们要求所有CCD评估,对于每个位移$p$,为我们提供足够准确的时间$t \in(0,1]$以满足正性。如果一对沿$p$发生撞击,我们需要一个时间t,以便缩放位移$tp$确保距离保持大于$\xi_k$,并尽可能接近目标分离距离$\xi_k + s(d_k^{cur} − \xi_k)$。

因此,使用有限厚度比不使用厚度更具挑战性。在接触 $d^{cur} − \xi_k$ 时,可以小至 $10^{−8 }m$(例如在大规模的压缩下),因此对于 $s\ge 10 ^{−9} m$ 左右的绝对误差是可以接受的。而在实践中,$\xi_k$通常处于$10^{−4} m$的尺度(例如,布料的厚度值为$\sim 3 \times 10^{−4}m$)。如果没有厚度 ($\xi_k = 0$),更新的距离以 $sd_k^{cur}$ 为目标(并且只需要严格大于 $0$)。换句话说,任何防止基元对交集的步长都是有效的,因此小于 100\% ≈ $10^{−9} m/s$ 的相对误差是可以接受的。另一方面,随着厚度的增加,更新的距离是针对$\xi_k + s(d_k^{cur} − \xi_k)$,因此,对于ξk的标准值,CCD评估的相对误差接近$0.01\%\approx 10^{ −9} m/(10 -4 m + s(10 −8 m))$,以避免相互穿透。

因此,对于现有方法来说,获得这种精度的CCD评估极具挑战性。作为起始示例,我们测试了Li等人的浮点CCD求解器,要求对两个具有挑战性的同维示例进行$s(d^{cur} − \xi_k)+ \xi_k$最小分离,厚度(见图10和21)$\xi_k>0$。在这里,即使我们完全删除保守的比例因子(即集合 $s = 0$),CCD 求解器也会错误地返回 $t = 0$ 影响时间 (TOI)(参见图 19)。注意,在执行此操作时,此错误会完全停止模拟进度。

\subsection{CCD 下界}

接下来,我们为CCD查询推导出一个有用的下限值,在数值上该下限值是鲁棒的,并且可以在浮点环境中有效地进行评估。这个下限提供了一个保守的、有保证的安全步长,也提供了一个明确的衡量标准来测试CCD查询的有效性:任何具有较小TOI的CCD类型评估显然都是失败的。在第 5.4 节中,我们应用此界限来推导出一个简单、有效且数值准确的显式 CCD 求解器,即使在薄材料仿真所需的具有挑战性的 CCD 评估中,该求解器也非常强大且高效。

在这里,在不损失一般性的情况下,我们将关注边对 $(x_0 ,x_1)$ 和 $(x_2,x_3)$ 之间的边-边情况,以及相应的位移 $p_0$、$p_1$、$p_2$ 和 $p_3$。任意点之间的距离函数分别由每条边上的$\gamma$和$\beta$参数化,则
\begin{equation}
  f(t, \gamma, \beta) = \| d(t, \gamma, \beta)\|
\end{equation}
然后,CCD评估寻求满足的最小正数实际值,满足以下条件:
\begin{equation}
  f_1(t) = \min_{\gamma,\beta} f(t, \gamma, \beta) = 0
\end{equation}
如果这样的$t$ 存在,我们称其为 $t_a$。参数 $(\gamma_a, \beta_ a) = \arg\min_{\gamma, \beta}f(t_a, \gamma, \beta)$依次给出在时间 $t_a$ 在两条边上碰撞的相应点。

我们可以直接精确地求解该问题。为了求解距离,已知$f_1(0)\le f(0, \gamma_a, \beta_a)$,那么三角不等式给出结果:
\begin{equation}
  t_a \ge \frac{f_1(0)}{\max(\|p_0\|, \|p_1\|) + 
  \max(\|p_2\|, \|p_3\|)}
\end{equation}
更广义的说,对于所有的查询,我们的下界$t_a$对应一个近似的距离函数,以及分母中两个基元中每个基元的最大位移之和。然后我们注意到,即使没有满足$f_1(t)= 0$的最小正时间(因此对区间没有影响),我们的边界仍然被明确定义为保守的步长。

接下来,我们观察到,也许令人惊讶的是,最先进的浮点CCD求解器可以并且将会返回小于我们下限的TOI结果,因此显然是错误的(见图19)。

最后,为了改善我们的边界,我们观察到它独立于参考系的选择而成立。因此,我们可以通过选择减少位移矢量$p_i$范数的帧来独立地进一步收紧每个 CCD 查询的界限。例如通过将每个$p_i$减去平均值 $\frac{1}{4}\sum_ip_i$。

\subsection{加性CCD}

使用我们计算的下限,我们现在应用 CA 策略为可变形体构建一种新的CCD算法,该算法具有有限偏移,该算法迭代更新并增加了我们在TOI的连续保守步骤中的下限。然后,所得的加性CCD(ACCD)方法可以鲁棒地求解有界TOI,单调接近每个CCD解,同时只需要显式调用来评估更新的基元位置之间的距离。

在每个 CCD 查询开始时,为了初始化 ACCD 算法(参见算法 1),我们首先将碰撞模板的位移居中在原点,以减少边界的分母,例如,$l_p = \max(\| p_0 \|, \| p_1 \|) + \max(\| p_2 \|,\|p_3 \|)$对于边缘-边缘对,因此增加我们可以安全采取的步长。如果没有相对运动($l_p = 0$),我们当然只是不返回碰撞,因此完整的单位步长是有效的。然后,我们根据当前的平方距离$d_{s q r}$和比例因子$s$计算$g = s(\sqrt{d_{s q r}} − \xi)$到偏移表面的最小间隔。为此,我们使用一个对取消错误更强大的公式。请参阅算法 1 中的第 8-9 行。
  
从最保守的冲击时间 $t = 0$(第 10 行)开始,我们创建节点位置的本地暂存器副本 $x_i$ ,并使用公式 10(第 11 行)初始化当前下限步长 $t_l$。

然后,我们进入迭代细化循环(第 12-21 行)以单调改进我们的 TOI 估计 t。在每次迭代中,我们使用当前步骤$t_l$(第13-14行)更新节点位置$x_i$的本地副本。如果这个新位置达到我们到偏移量的目标距离表面(变得小于g)我们已经收敛,前一个$t$是撞击时间,使距离刚好达到$g$(第17行)。如果没有,我们将当前 $t_l$ 添加到 $t$(第 18 行)来更新我们的 TOI 估计值。请注意,我们总是将第一个下限步长添加到 $t$(第 16 行),因为它保证不会使距离更接近 $g$ 。如果我们的TOI现在大于$t_c$,当前最小首次撞击时间(或者可以简单地设置为1),我们可以不返回碰撞(第19-20行)。否则,我们从更新的配置(第 21 行)计算一个新的局部下限 $t_l$(以 0.9 缩放以改善收敛性),并开始下一次迭代。

因此,ACCD提供了一个极其简单易实现、数值稳健的CCD评估。它只需要显式调用距离评估,因此没有数值上具有挑战性的寻根操作。反过来,ACCD能够支持厚度偏移和可控的精度,因此在CCD应用中灵活地调整性能与精度权衡。在第 6.6 节中,我们将 ACCD 与最先进的 CCD 求解器进行了比较,在那里我们看到它们都严重失败,导致交叉或微小的 TOI,从而在具有挑战性的示例中阻碍优化。反过来,在替代CCD方法成功的情况下,我们看到ACCD实现了类似且通常改进的计时性能。最后,我们指出,ACCD的停止标准要求$s > 0$(我们在所有示例中都应用$s = 0.1$)来提供有限终止。反过来,ACCD的目标不是计算精确的TOI,而是(如接触处理应用中实际要求的那样)获得可靠的,无交集的TOI步骤。

最坏情况的性能。如上所述,我们发现在实践中,特别是在我们所有具有挑战性的测试中,ACCD与稳定的性能(其他方法成功时相当或更快的时间,所有其他方法失败时的有效时间)收敛。然而,值得分析的是,应该有可能有无限的、最坏情况的表现。回想一下,我们选择参考系,使位移总和为 0。但是,如果一个基元有一个无法抵消的发散位移场,我们的分母l p可以保持很大。然后,如果基元的起始距离 $d$ 也很小,因此分子也相应很小,则在这种情况下 ACCD 的迭代计数可能会很大。虽然我们看到这种情况在实践中不会发生弹性动力学,但扩展ACCD以加速这些可能情况的收敛,从而获得有限的性能保证是一个有趣的未来步骤。





\section{性能测试}

我们在C++中实现了我们的方法,应用了 CHOLMOD [Chen 等人,2008 年],使用英特尔 MKL LAPACK 和 BLAS 编译用于线性求解,用特征编译用于剩余的线性代数例程 [Guennebaud 等人,2010]。C-IPC的必要导数以及我们应用于效率和数值鲁棒性的代数简化在我们的补充文档中有详细说明。为了实现未来的应用、开发和测试,我们将把C-IPC的实现作为一个开源项目发布。我们所有的实验和评估都是在英特尔 16 核 i9-9900K CPU 3.6GHz × 8(32GB 内存)、英特尔酷睿 i7-8700K CPU @ 3.7GHz × 6(64GB 内存)或英特尔酷睿 i7-9700K CPU @ 3.6GHz × 4(32GB 内存)上执行的,详见下文实验。

\textit{实验}:下面我们从一项研究开始,说明和分析标准布料和网眼的膜锁定行为(第 6.1 节)。然后,我们在6.2节中展示了C-IPC在完全耦合所有物理力的同时严格满足应变极限的能力。据我们所知,C-IPC是第一种既能严格满足应变极限,又能支持极限、弹性和接触力完全耦合的方法。为了分析应变极限满足和完全耦合的影响,我们接下来考虑与先前方法的比较。如第2节所述,应变限制中的现有方法会产生包括锁定、抖动和互穿在内的伪影。这些问题有两个来源:1)应用的分裂模型不准确,2)无法满足计算中的应变极限。在这里,我们首次分别分析了与求解器精度误差无关的拆分模型(第 6.3 节)所产生的问题。然后在第6.4节中,我们考虑了由于分裂误差和方法无法执行要求的应变限值而导致的伪影和不准确性(由最先进的布料代码产生)。在6.5和6.6节中我们分析了C-IPC的厚度建模,与现有的布料模拟规范和以前的CCD评估方法进行了比较。最后,在评估了C-IPC的所有组件后,我们评估了C-IPC在服装模拟中的应用(第6.7节)及其对先前压力测试布料基准的解析(第6.8节)。然后,我们在具有一致厚度建模的任意余维全耦合系统的仿真中演示了C-IPC(第6.9节),并最终考虑了其在一组旨在锻炼鲁棒性和准确性的新布料模拟应力测试中的性能(第6.10节)。所有实验设置和统计数据都列在图 24 中。我们报告系统中涉及的节点总数,包括运动学(相等约束)的节点。所有示例都直接应用 IPC 的平滑半隐式摩擦(1 次滞后迭代)和默认牛顿公差(如果未另行提及)。

\subsection{布料材质}

在这里,我们首先说明并检查膜锁定对现实世界材料的影响,以及应变限制减轻它们的能力。在Penava等人[2014]中,测量并验证了一系列布料的密度,厚度和方向依赖性的膜应力 - 应变曲线。然而,在具有标准分辨率网格的仿真中直接应用这些真实世界的布料参数会产生严重的膜锁定。这是不可避免的,除非使用非常高且经常不切实际的大网格尺寸。在这里,我们检查这种锁定行为,首先使用我们的IPC模型进行演示,没有应变限制。我们还注意到,这种锁定行为与算法无关,因此,例如,也很容易在其他代码中演示,例如 ARCSim [Narain 等人,2012 年],以及它们在现实世界中捕获的材料参数 [Wang 等人,2011 年];参见图 15d。

我们首先考虑一个简单的1m×1m非结构化网格在固定球体和地平面上掉落的行为(所有摩擦力为$\mu = 0.4$)。我们应用Penava等人[2014]测量的棉布密度($472.6kg/m^3$)和厚度(0.318mm),然后考虑改变膜刚度,同时保持弯曲杨氏模量为0.8MPa,两个膜和弯曲的测量值为经纱方向的测量值为0.243。特别是Penava等人[2014],发现膜杨氏模量范围从0.8MPa到32.6MPa,用于不同的平面内方向。

对于我们的示例,我们发现,即使应用各向同性膜模型,具有最小的确定膜刚度(0.8MPa),我们也观察到严重的锁定伪影(参见图11a),其中弯曲是人为硬化的。如果我们接下来将这个最小的测量膜刚度降低0.1×,那么我们会看到人工弯曲伪影在很大程度上被消除,但我们仍然获得由主导膜能量强迫的尖锐折痕伪影(参见图11b)。接下来,如果我们尝试更小的0.01×膜刚度缩放,可观察到的膜锁定效应现在已经消失,但正如预期的那样,得到的材料被拉伸得太过长,因此甚至没有接近所需的材料行为(图11c)。这个简单的例子很好地演示了在模拟刚性真实世界布料时膜锁定的挑战。然后我们注意到,这些伪影只会在更具挑战性的模拟中加剧,例如移动边界条件和紧密接触。

接下来,我们使用C-IPC的各向同性模型应用严格的应变限制,以将应变限制在Penava等人[2014]测量的弹性范围内。棉花在各个方向上的最宽范围允许拉伸系数高达 6.08\%。在这里,应用此边界,即使膜刚度缩放到0.01×,我们现在重新获得没有膜锁定的模拟,并且由于仅限于测量的应变极限,也没有不自然的拉伸伪影(图11d)。

最后,需要重申的是,膜锁定确实是一种与分辨率相关的伪影,其影响会随着网格尺寸的增加而减小。例如,在 1× 节点的网格下,覆盖在球体上的布的锁定伪影在 85K 节点时弯曲刚度显着降低,而在 246K 节点处则难以察觉。但是,这种改进在很大程度上取决于场景参数。例如,在完全相同的场景中,相同的246K网格表现出明显的锁定伪像,如果我们只是将弯曲刚度降低到以下方面。

0.01×与图6底行应变极限模拟中的材料相匹配。请参阅我们的补充文件,了解这些案例的详细信息和说明。

\subsection{精确应变限制}

以上我们已经证明了应用应变限制在模拟布料材料中的重要性和影响。在这里,我们研究了C-IPC在我们的各向同性和各向异性模型中准确执行严格应变极限的能力。

\textit{在降低应变极限的范围内保持精度}:为了确认 C-IPC 本构应变极限的准确性、可控性和鲁棒性,我们测试了越来越严格的 1\% 和 0.1\% 应变极限(远低于标准布料中的测量极限)。我们将它们应用于上一节中考虑的球体悬垂示例以及使用两角钉布进行的简单膜锁定测试[Chen等人,2019;金等人,2017]。在这里,应用 C-IPC,参见图 12,我们观察到稳定的仿真输出,所有三角形上的拉伸都满足其规定的限制约束。仔细检查图 12 中的悬垂球体测试也证实两者都不存在人为弯曲刚度。相比之下,将这些结果与图11a中的人工硬化进行比较,尽管直接应用了测量的膜刚度,但最大拉伸超过4.5\%。

在这里,当我们单方面应用应变限制时,这与Jin等人[2017]的论点一致,即单边应变限制可以比双边执法更好地避免膜锁定 - 至少在一定程度上。如果限制太严格,即使是单边执法也不是完美的灵丹妙药。举一个极端的例子,我们可以将应变极限提高到一个非常小且基本上不切实际(0.1\%)的应变极限。这提供了一个极端的压力测试,我们确认C-IPC确实在所有时间步长中保持了这些具有挑战性的限制。然而,正如预期的那样,我们最终可以在图12b中观察到一些轻微但明显的锐利边缘折痕伪像,在图12d中观察到类似可见的锁定问题。尽管这种极端极限在实践中不太可能发生,而且布料通常也不会遇到,但该实验确实强调了一个重要的观点:如果大多数单侧应变极限约束是有效的,那么它们的行为与双边情况相似。因此,当大多数应变极限约束处于绑定活动约束数时,可以再次接近自由度的数量,因此,如Chen等人[2019]所述,可以再次遇到锁定。在这里,我们专注于制定一个可控且鲁棒的应变极限本构模型。我们希望这将通过探索应变约束的交替离散化来进一步增加无锁定配置的范围,如Chen等人[2019]。但是,我们注意到,当此处提出的底层方法能够准确地保证耦合的约束满足时,可以最好地启用此功能。锐利边缘折痕伪像,在图12d中观察到类似可见的锁定问题。虽然如此极端的极限

\textit{各向异性应变限制}:对于各向异性,情况类似。在图 13 中,我们演示了 C-IPC 的各向异性模型,包括球体悬垂和双角悬挂测试。我们再次观察到,应用Clyde测量的刚度值会产生清晰的膜锁定伪影(见图13a)。接下来在图13b和c中,我们分别将膜刚度缩小0.1×和0.01×。在整个过程中,我们确认C-IPC执行了克莱德测量的应变极限,并再次获得了预期的改进结果,既没有锁定伪影,也没有非物理拉伸,刚度为测量刚度的0.01×。同样,在图13d中(同样是0.01×报告的刚度),我们应用C-IPC来保持测量的应变极限,使用各向异性模型获得无伪影的皱纹。

\subsection{对比算符分裂模型}

现有的应变限制方法引入了来自两个主要来源的误差。首先,在建模级别,他们将应变极限与弹性分辨率分开。其次,无论采用何种分裂模型,用于求解它们的算法都是不准确的,并且通常无法满足甚至中等要求的应变极限,正如我们将在下一节中看到的那样。然后,所有先前的方法都会引入来自这两个来源的错误,因此尚不清楚每个来源都产生了哪些问题。在这里,我们首先分别分析拆分模型产生的问题,通过精确地解决拆分的每一步。这使我们能够证明,无论如何准确地执行极限约束,拆分本身都会引入不可避免的错误。然后,在6.4节的下一步中,我们将检查求解精度,并看到应变极限求解本身中的误差会产生不一致且通常无法控制的仿真结果。

为了检查分裂误差,我们首先应用标准的应变限制,分裂策略,因此将每个时间步长求解分为两个连续步骤。第一个求解一个预测器步骤,该步骤包括整个系统的所有力,除了应变限制,以获得中间配置$\hat x$。接下来,第二步预测预测变量以满足应变限制。为此,我们最小化本构应变极限能量,总和为从最终时间步长解到$\hat x$的质量加权$L^2$距离。

在最简单的情况下,我们可以考虑在没有接触力的情况下这种分裂的影响。我们从固定和拉动的方布开始。我们固定它的两个顶角,并施加重物将其两个底角向下拉;参见图 14a 和 b。相比之下,在没有应变限制的情况下,布被拉伸超过10×,而在我们的本构应变极限下,应变被限制在测量的弹性范围内,在这个完全耦合的解决方案中,我们看到预期的垂直对齐的皱纹在两个底部拉角处拉伸;参见图 14a。另一方面,在图14b中,我们看到弹性求解的分裂应变限制在皱纹在非物理方向对齐的两个底角产生明显的偏置伪影。这些误差是由应变限制力的解耦引起的,随着时间步长的增加而变得越来越严重(这里我们用$h = 0.04s$)。

接下来,为了考虑接触的分割误差,我们再次考虑球体上的布料示例。在这里,我们应用新胡克膜弹性来帮助分裂方法避免可能的三角形简并。请注意,即使在这里,我们的完全耦合的C-IPC模型实际上也不需要新胡克弹性(因为三角形退化是通过邻居之间的点三角形约束来防止的)。然而,为了保持一致的比较,我们对两者应用了Neo-Hookean模型。在图14d中,我们看到接触的两步分裂现在受到严重的压缩伪影的影响,这再次来自第一步的弹性动力学解析,然后在第二步中施加应变限制和接触力。为了进行比较,请考虑图14c中C-IPC相应的全耦合应变极限解决方案。反过来,分割解中的误差表明,应变的额外下限(例如,在 ARCSim 中应用的应变)可能有助于避免这些压缩误差。如果我们另外添加一个应变极限下限(此处为 0.7),那么我们确实会发现得到的分裂解决方案现在没有严重的压缩伪影。然而,现在,由于膜和弯曲的错误,分裂溶液中的布仍然不自然地平放在地板上;参见图 14e。

当然,与所有时间步长拆分方法一样,可以通过应用越来越小的时间步长来减少拆分误差。例如,在这里,我们发现分割模型和完全耦合求解之间的视觉上明显的误差在$h = 0.004s$时消失了。但是,正如预期的那样,计算时间的大幅且通常不可接受的增加抹去了拆分的任何预期收益(我们将在下一节与现有布料代码的比较中再次看到此主题)。同样,时间步长的必要减小随示例和场景而变化,因此似乎不可能实现具有拆分功能的稳健、可控的时间步长。

\subsection{对比ARCSim的应变限制求解器}

上面我们分析了应变限制的时间步长分裂产生的建模误差。为此,我们比较了每个阶段在通用基准代码中始终如一地求解的模型。接下来,我们考虑拆分误差与不准确的约束求解相结合在现有布料仿真代码中产生的联合影响。为此,我们与ARCSim进行了比较,据我们所知,这是目前最强大的外壳模拟器,可以支持真实世界布料材料参数的应变极限。有关 ARCSim 和 C-IPC 实现的应变极限和时序的统计数据,请参见图 16。

我们再次从考虑更简单、无接触的情况开始。为此,我们考虑一个更简单的布挂示例:通过放下没有加重底角的固定布。我们应用 ARCSim 的默认棉材料和默认应变极限设置,这些设置限制在$ [0.9, 1.1] $范围内。在这里,即使在如此广泛的允许应变和全膜刚度下,ARCSim结果在每个时间步都明显违反了极限界限。为了直观地参考图15a,我们说明了满足应变极限界限的平衡下的完全耦合C-IPC解决方案。与图15b中平衡时的ARCSim输出进行比较,我们很容易看到固定角附近过度拉伸形成较大弧线的伪影。这里两帧都是在 4s 拍摄的,并且时间步长为 $h = 0.04s$。

当我们将 ARCSim 的时间步长减小到 $h = 0.01s$ 时,这些误差仍然很大,直到我们达到 $h = 0.001s$ 的步长时,应变极限才基本(尽管仍然不完全)得到满足。然而,要做到这一点,ARCSim需要87分钟来模拟这个简单的4s仿真序列,与C-IPC的完全耦合、满足应变极限的解决方案(在h = 0.04s时步进)相比,速度减慢了10×以上。然而,除了必要的减速之外,每个场景和材质所需的步长减小变化,因此总是不清楚,如果没有每个场景进行许多耗时的模拟测试,如何确定必要的时间步长减少,以确保 ARCSim 结果满足规定的应变限值。相比之下,C-IPC再次在时间步长、材料和场景设置方面保持应变极限约束满足。为了进一步研究ARSim的约束误差,我们还观察到ARCSim中使用的增强拉格朗日应变限制求解器应用了设置为100的硬编码最大迭代上限。通过实验增加这个极限(否则保持ARCSim代码不变),我们确认即使将这个上限增加一个数量级,应变极限仍然没有达到满足。

接下来,我们为 ARCSim 比较添加联系人,并考虑 8K 节点球体悬垂测试。在这里,我们再次观察到,应用应变限制来减少膜锁定是行不通的,相反,随着膜硬度的降低,应变极限满意度会恶化,伪影实际上会增加。

具体来说,我们首先应用 ARCSim 的默认棉质材料和与上述相同的默认应变极限设置。对于这种刚性材料,我们观察到预期的膜锁定问题;参见图 15d。然而,我们确实注意到,对于这种刚性膜,除了几个时间步长之外,应变极限约束肯定得到了很好的满足。

接下来,作为标准,我们降低 ARCSim 中的膜刚度,希望减轻锁定,并期望应变限制将补偿。如果我们将拉伸刚度降低 10×,锁定伪影确实更少;但是,ARCSim的应变限制不能为刚度降低提供预期的补偿。相反,在整个仿真过程中,违反应变极限的三角形明显更多(见图15e),导致拉伸伪影。如果我们进一步将刚度降低到默认棉布值的0.01×(在本例中通常是去除所有可见锁定伪影所必需的;例如,与我们的可比刚度进行比较,C-IPC结果如图11d和15c)ARCSim在图15f中的结果显示更明显地违反了应变极限,因此遭受了与上一节中分裂模型分析相同的极端压缩问题;有关比较,请参见图 14e。此外,我们现在还观察到 ARCSim 进一步的三向分裂引起的抖动,该分裂进一步将应变极限求解和接触力分成两个独立的顺序投影步骤。我们注意到,最后一个问题在我们的分析中并不陌生,Narain 等人对此进行了讨论。

\subsection{厚度建模}

接下来,我们研究接触中小而有限的厚度建模。为了说明C-IPC模型与现有最先进的外壳模拟器之间的差异,我们检查了一个简单但具有挑战性的布料堆叠基准。我们构建了一个场景,其中堆叠了十个 8K 节点的方形布面板,这些面板以垂直间距和旋转方向对齐,同时放在固定的方形板上。有关初始设置,请参见图 17a。这应该形成一个布堆,当我们改变材料厚度时,整体桩高和体积特性也应相应变化。

\textbf{ARCSim} [Narain et al. 2012] 计算了 3 个碰撞响应,应用 Harmon 等人 [2008] 的非刚性撞击区和应用于模型布厚度和稳定碰撞处理约束的排斥厚度。ARCSim的厚度参数默认值设置为5mm,并显示为可由用户更改的配置。然后在代码中将此参数与直接应用于碰撞投影、分析和接触力计算的其他厚度参数相关联。从默认的 5mm 厚度设置开始,以越来越小时的时间步长进行测试,低至 h = 0.001s,ARCSim 始终报告碰撞处理失败,并在仿真结果中表现出严重的伪影;参见图 17b。接下来,应用10mm厚度的ARCSim报告在h = 0.001s时成功解析接触,但仍会在几何和动力学中产生较大的伪影;参见图 17c。我们没有将 ARCSim 进一步推到小于 h = 0.001s 的时间步长,因为这已经使 ARCSim 的计算速度过慢;例如,与C-IPC相比。

\textbf{ARGUS} Li等人更新了 ARCSim,改进了用于壳体仿真的接触和摩擦处理。与 ARCSim 一样,它也应用并向用户公开相同的厚度参数。使用默认求解器参数,在h = 0.001s时,ARGUS能够模拟默认5mm材料厚度(图17e)和较厚的10mm材料(图17f)的布料堆叠。在这里,我们还观察到桩中的一些高度差被捕获。然而,当我们减少厚度,使更薄、更标准的服装材料厚度减少时,例如将厚度设置为1mm,ARGUS会产生严重的伪影,如图17d所示。与ARCSim一样,对于所有三种厚度设置,ARGUS的计时仍然比C-IPC慢得多。

总之,我们观察到,最先进的方法需要很小且通常不切实际的时间步长,以避免随着建模几何厚度的减小而失败。随着厚度的减小和/或碰撞速度的增加,避免故障所需的时间步长会减少,从而相应地增加仿真成本,并且不清楚每个场景和厚度成功所需的时间步长。

\textbf{C-IPC} 对于相同的厚度基准,只需将$\hat d$(回想一下我们开始施加接触力的距离)设置为有效厚度值1mm,5mm和10mm,即可始终对不同的厚度进行建模。在图18的第一行中,C-IPC连续模拟所有不同的材料厚度和相应的布料堆叠高度。然后,C-IPC的模型进一步提供了法线方向上厚度的本构行为。图 18 的中间行和底部行对此进行了演示,然后我们将一个弹性球放在堆栈上。在中间一行中,我们显示了每个堆栈碰撞的最大压缩,观察随着厚度的增加而增加的压痕。另请注意,皱纹仅在最厚的 10 毫米外壳中形成。在底行中,我们显示了碰撞后每次模拟的最终平衡帧,说明了由体积有限元球模型加权的壳堆的不同静止高度。这个例子还说明了C-IPC对不同弹性模型的自然耦合,我们将在第6.9节中详细探讨。这些示例还有助于说明 C-IPC 的计算成本如何随阈值 $\hat d$ 而变化;参见图 24。一般来说,我们看到在较大的阈值(例如$\hat d$ = 10mm)下,C-IPC每步处理更多的触点对,使得仿真比在较小值下更昂贵,每次求解的触点更少,例如$\hat d$在1mm和5mm处。同时将d设置为小几个数量级的值,例如1 $\mu m$,由于势垒函数的清晰度增加,收敛速度较慢,因此计算时间更长。

\subsection{CCD 评测}

对于最后一部分的比较,仅通过d模拟C-IPC中的厚度效应就足够了,因此纯粹是一种弹性行为。更一般地说,如第5节所述,随着厚度变小,接触更紧密,需要同时使用$d$和偏移$s$的C-IPC非弹性厚度模型。

为了模拟共维对象,这两种机制结合在一起。在这里,较大的 $\hat d$ 引入了增加的弹性海绵,而非零 $s$ 即使在极端压缩下也能保证最小厚度;例如,参见图2。对于厚度一致建模很重要的场景;见例如。在图2、3、10、21、22和23中,我们再将s设置为所报告的物体的材料厚度值或稍小,然后在$t$附近设置$d$,以控制所需的弹性响应程度。

如第5.3节所述,仿真这些示例会产生更多的简并接触对,并且需要更高的精度来捕获厚度偏移,因此对CCD精度提出了很高的要求。在这里,IPC的保守 CCD 策略已不再足够。使用标准浮点CCD例程可以并且将简单地返回许多查询的TOI,因此会错误地完全停止IPC优化进度。同样,在大多数情况下,具有更高报告精度的替代CCD方法同样失败。

这激励了我们新的ACCD方法,如下图所示,该方法仍然稳健而准确 - 即使在我们所有可用的CCD替代品都失败的最复杂的例子中,也能在IPC优化方面取得稳定,向前的进展。与此同时,ACCD继续在替代CCD方法能够成功的较简单例子上提供可比且通常更快的时序。

为了与ACCD进行比较,我们测试了一系列CCD方法:最小分离(MinSep)CCD 、根奇偶校验、BSC、T-BSC和 IPC 的保守浮点 CCD,一组十个具有具有挑战性的同维碰撞的基准示例。基准测试中的前五个示例具有厚度偏移,其余五个示例没有;参见图 19。

我们能够通过在与IPC相同的保守CCD策略中将其用作基础求解器来直接测试MinSep。此处的最小分离设置为$s \times d^{cur}$,当前比例因子 $s$ 和距离 $d^{cur}$ 。如果返回的 TOI 小于 $10^{−6}$,则再次执行不缩放($s = 0$)的查询,以尝试使查询更容易。为了测试根奇偶校验、BSC 和 T-BSC 方法,它们只决定一个区间是“有冲突”还是“没有冲突”而不计算 TOI,我们对每个查询应用它们而不回溯,直到找到没有冲突的步长。找到后,我们返回步长的保守缩放 $1 − s$。由于回溯不能直接将这些方法扩展到计算首先将接触对的距离带到ξ的时间或步长,因此我们无法在具有厚度偏移的示例上测试这三种方法;因此,我们将对它们的测试限制在图 19 中的最后五个测试示例中)。

我们在图 19 中总结了我们的比较。IPC使用浮点CCD的保守策略在更简单的场景中效果很好,尽管时序比我们的ACCD方法稍慢,但在更复杂的场景中却失败了。我们还注意到,虽然保守的浮点CCD通过查询额外的点-边和点-点对来处理平行边-边等退化情况,但ACCD只需要查询一般对(例如,仅表面场景的点三角形和边-边对)。另一方面,MinSep 总是为所有十个基准示例返回微小的值,有时甚至是 0,因此没有任何进展。根奇偶校验在它可以应用于的大多数基准示例(没有厚度偏移的基准示例)上表现良好,效率很高,但不幸的是,它完全错过了一些碰撞,导致针床和服装等示例中不可接受的相互渗透。对于 BSC,我们发现它报告了所有基准示例的运行时错误,可以使用内部消息“拐点在 BSC 中未完全处理”来应用,而 T-BSC 返回服装示例的微小值,并且无法在没有碰撞的情况下确定步长(我们确认两种 BSC 变体在简单情况下都运行良好),因此在它可以应用于的其他五个基准示例中陷入行搜索。相比之下,我们看到ACCD使所有示例能够完成收敛,而对于替代CCD例程能够完成的示例,它提供了最快的时间。最后,作为对C-IPC框架之外的ACCD应用的初步调查,我们应用ACCD代替阿格斯默认的CCD例程,并使用1mm和5mm布堆叠示例进行测试。虽然正如预期的那样,仿真结果几乎相同(没有视觉差异),但CCD成本降低了约50\%。

\subsection{服装模拟}

接下来,我们应用C-IPC来模拟服装的悬垂和动态。

阿格斯比较。我们首先考虑C-IPC在ARGUS提出的一对测试舞蹈序列。这两个示例都使用了Li等人中测试的最高分辨率输入网格。在这里,杵状指练习接触处理更广泛,动作和过渡比蔓藤花纹更快。为了获得可比较的结果,我们在同一台机器上运行C-IPC并重新运行ARGUS,ARGUS的网格分辨率固定。在这里,我们在有和没有应变限制的C-IPC的整个序列中观察到稳健、平滑的仿真结果。值得注意的是,即使增加了严格的非交叉执行、收敛、完全隐式的时间步长和解析所有接触模板,而不是像 ARGUS 处理中那样仅解决节点节点,我们仍然观察到 C-IPC 在蔓藤花纹和 Clubbing 上的加速比 ARGUS 结果高 3.7 倍和 1.5 倍。同样,如果我们也启用C-IPC的本构应变限制,加速比同样保持在2.7X和1.4X。有关比较时序,请参见图 20,有关结果,请参见我们的补充视频。

\textit{服装模拟挑战}:服装仿真中的两个主要挑战是解决复杂的多层设计,以及模拟快速移动的角色所穿的服装。为了测试C-IPC在这些情况下的鲁棒性,我们构建了两个新示例。由于其复杂的缝合图案,这些服装无法在ARGUS和ARCSim代码中模拟。

我们模拟了 FoldSketch 制作的黄色连衣裙(15.7K 节点),其中创建了难以解决的刀褶,其中布料折叠在自身上,然后在接触时缝合紧。我们以围绕 12.8K 节点人体模型的平面模式开始 C-IPC(图 5a)。然后,我们通过向前推进CIPC来悬垂,拼接力(弹簧)以大时间步长将设计缝合在一起。C-IPC在经过几个大的时间步长后获得垂坠服装的静态平衡,获得锋利的刀褶,同时保持正确的分层顺序,没有交集(图5b)。然后我们将刀褶通过伦巴舞的台阶,保留了带有刀褶的连衣裙的流畅运动,保持清脆并在整个过程中适当碰撞(图5c)。

然后,我们应用C-IPC来类似地初始化使用敏感时装设计的多层裙子图案(30.5K节点)[Umetani等人,2011]。这得到了图5d中的初始姿势静态悬垂。接下来,我们编写穿着分层服装的人体模型网格,通过快速移动的武术序列 4 以大动作。在图5的底行中,我们显示了生成的C-IPC仿真的三帧,其中包含快速踢腿序列的一部分。另请参阅我们的补充视频。C-IPC稳定地解决了这些快速运动,以h = 0.04s的大时间步长继续,同时捕获复杂的服装细节,在整个过程中执行应变限制和非交集(从而保持分层顺序)。最后,我们将相同的服装通过较慢的伦巴序列,并观察到成本相应降低,因为需要更少的迭代来解决具有更温和运动和相应减少接触次数的时间步长。

\subsection{布料模拟基准测试}
在本节中,我们确认了C-IPC在先前工作中具有挑战性的基准测试中的表现。

旋转球体上的布。在图6中,我们测试了Bridson等人[2002]设计的旋转球型示例,这些示例旨在进行摩擦,接触处理和紧密起皱。我们首先将一块 86K 节点方布($\mu$ = 0.01 和 1.0608 应变上限的非结构化网格)放到球体和地面上(均为 $\mu$ = 0.4)。当球体旋转时,摩擦力会抓住布料,当布料向内拉动时,会捕捉到细小的起皱,而不会造成锁定和拉伸伪影。我们确认,随着我们提高分辨率(至246K节点)并降低弯曲刚度(降低10×),更高频率的皱纹在褶皱中消退。

漏斗。为了测试分层、摩擦和与应变限制的紧密接触,我们扩展了 Tang 等人 [2018] 和 Harmon 等人 [2008] 的漏斗示例。在图 7 中,我们将三块布板($\mu$ = 0.4,每个 26K 节点,应变上限为 1.0608)放在漏斗上。在摩擦下,它们稳定地放在顶部,直到一个尖锐的、脚本化的碰撞物体(四节点四面体)迅速将面板向下推并通过漏斗。C-IPC求解序列中的所有时间步长以收敛,同时始终保持无交集,应变极限满足轨迹。

礁结。为了将极端压缩,接触和摩擦与大应力相结合,我们匹配了Harmon等人[2009]的礁结示例的初始设置。两条丝带最初交织在一起,然后拉伸形成图 4 中的结。在这里,我们通过模拟100K节点(原始节点的10×)来扩展原始测试的挑战,施加1.134的应变极限,摩擦$\mu$ = 0.02,并拉得更紧,将带拉伸到接近其应变极限。这捕获了轨迹的摩擦动力学(没有抖动),并在中心处产生极端应力,形成比最初显示的要小得多的结。

\subsection{壳体、棒、颗粒的耦合}

在本节中,我们将演示跨所有协域的C-IPC仿真。这包括体积有限元材料、壳体、棒材,甚至在需要时还包括颗粒。由于这些域通过IPC屏障相互作用,因此我们能够在同一仿真中无缝且直接地耦合所有这些基于网格的模型。这些测试还突出了我们一致且可控的厚度建模对这些薄材料相互作用的优势和适用性。

\textit{洗牌}。我们在复杂的交错洗牌过程中测试非常薄的壳体摩擦接触的准确性和精度。我们从 54 张硬壳扑克牌(63.5 毫米× 88.8 毫米)开始,IPC 偏移和d分别设置为 0.1 毫米和 0.2 毫米,以解析卡的厚度。在图 8a 中,我们将卡片分成两堆,并在它们的侧面应用移动边界条件来弯曲它们,为“精确”桥洗牌做准备。与人类表演的桥牌完成不同,卡片开始交错,在这里我们通过保持两个桥桩分开来增加挑战。然后,我们从下往上逐张卡片释放保持的边界条件,在左右侧桩之间交替(图 8b)。释放的牌迅速恢复平坦的休息形状并落在地上,形成一个新的、洗牌的、无交叉的堆叠堆。然后,我们用两个脚本碰撞边界(图 8c)对甲板进行部分平方,得到一个 15.4 毫米高的最终桩,并且与真实甲板(图 8d)。在这里,我们应用了0.5×的默认公差来捕捉复杂的动态。

\textit{毛}:通过自碰撞对头发进行建模是杆接触加工的严峻挑战 - 部分原因是它们的横截面极小。我们在两个具有挑战性的头发模拟测试中检查了C-IPC的行为。头发厚度很小,但其体积效应很重要,并且仍然难以捕捉。在这里,C-IPC通过对所有$\hat d = 0.1mm$的杆强制进行0.08mm的偏移来捕获头发厚度。在图3(a)-(e)中,我们首先用编织测试C-IPC。我们通过相互扭曲两簇头发的底部自由度来形成一个紧密的辫子。尽管承受了很大的应力,但产生的编织物仍然没有相交,并在整个过程中保持厚度偏移。我们还通过释放底部保持边界条件并让辫子松开而不会造成伪影或不稳定因素来在动力学中证实这一点。接下来,我们通过将一簇具有一个自由端的头发放在类似的垂直悬挂集群上来进一步测试头发自碰撞和摩擦[McAdams等人,2009]和图3(f)-(k)中的固定球体。C-IPC再次模拟了这个场景,并在整个过程中进行了收敛的大时间步长求解。

\textit{面条}:一大碗面条很美味,但由于大量的自由度和高接触次数,模拟起来也很困难。在图 10a 中,我们将一堆 625 根 40 厘米长的煮熟的面条放入直径 13.85 厘米的碗中(固定几何形状)。每根面条都是一根 200 段离散的杆,$\mu$ = 0.4,偏移和 $\hat d$  分别设置为 1mm 和 0.5mm,以模拟面条的 1.5mm 厚度。随着C-IPC的厚度处理,掉落的杆停止(它们的总体积应为4.42×10 −4 m 3),主要填充碗(图10b),其体积为6.96×10 −4 m 3。在平衡时 在图10c中,我们移除碗以显示最终交织,无相互渗透的配置。最后,在图 10d 中,我们删除了体积渲染,以突出显示这个复杂的非相交体积仅使用离散杆折线进行模拟。

洒。同样,我们可以只对边缘段做同样的事情。我们将一个由 25K “洒”组成的网格,每个“洒”由 6 毫米长的单边缘段形成到一个碗中(图 21a)。洒水的 2mm 厚度由 1.5mm 的 C-IPC 偏移和 0.5mm 的 d 建模。使偏移量比 d 大 3× 模型是其碰撞的强非弹性响应。与面条一样,我们可以估计洒水的总预期体积为 $5.76 \times 10^{−4} m^3$,在平衡时填充碗的 $6.88 \times 10^{−4} m^3$ 体积,没有相互渗透(图 21b 和 c)。同样在图 21d 中,我们删除了体积渲染以突出显示仅使用同维线模拟的几何图形段。对于面条和撒布,我们使贴合的圆柱形表面比$\xi + d$ 薄 0.2mm。

布上的球形颗粒。在极端情况下,C-IPC甚至可以通过将单个顶点视为具有严格厚度偏移的颗粒来模拟小球形颗粒。我们通过将一列 50K 粒子滴到带有固定角的布上来测试非相交粒子轨迹的分辨率(图 22)。即使在高速行驶和与布料下,碰撞也不会相交。静止颗粒捕获紧密的球形填料(参见放大图),确认偏移保留。一旦颗粒堆积在中心,我们就会释放一个布角,让它们流向地面。为了模拟每个粒子,我们计算质量$\rho= 1600kg/m^3$和$V = \frac{4}{3}πr^3 \approx 4.2×10 −9 m^3$,并设置厚度,C-IPC偏移为2mm,$\hat d$为1mm。以这种方式模拟粒子类似于离散元法(DEM)[Cundall and Strack 1979;Yue 等人,2018 年],其中粒子被“涂覆”有相互作用区(在我们的设置中为 $\xi < \hat d < \xi + \hat d$),用于根据区域交点的深度、面积或体积确定接触力 [Jiang 等人,2020]。DEM是地质力学中的关键工具,它通过忽略粒子变形来提供合理的简化,同时准确捕获动量和粒子间的接触力。在这里,C-IPC通过我们的屏障功能将接触力直接与距离联系起来,因此可以与可变形材料无缝耦合,就像这里的布料一样。与DEM不同,C-IPC还额外保证了由偏移定义的粒子体积的无相交轨迹。

全包。最后,我们测试将所有余维耦合在一起。我们将FE犰狳体积,贝布和颗粒柱放到由交错杆形成的网上(图23a),对所有域施加0.5mm的C-IPC偏移和1mm的$\hat d$。我们一次依次删除这些共域。首先,掉落的犰狳被缠住并夹在网的杆之间(图23b)。接下来,布落在犰狳上(图23c),最后柱子倒在布上,流入由犰狳和杆网支撑的两个布褶皱中(图23d,e和f(俯视图))。

\subsection{布料仿真的新测试}

最后,我们在一组三个新的、具有挑战性的布料仿真实例上提出并评估了C-IPC,这些仿真示例在模拟具有摩擦接触、应变限制和厚度的弹性动力学时需要极高的鲁棒性和准确性。据我们所知,C-IPC是实现这些场景和行为模拟的第一种方法。

\textit{在同尺寸针上拉布}。众所周知,针对尖锐几何形状的布接触模拟是应力接触模拟,包括钩住和落下,然后将布板拖过由线段形成的针床(图 9)。由于$h = 0.02s$的大时间步长和 1.0608 的应变极限,布料可以安全地静止在针床上而不会抖动。我们以 1m/s 的速度拉过段尖,观察沿着针床顶部快速滑动,然后离开针床顶部,没有钩住或拉伸伪影(整个应变限制都满足)。同样,如果这些接触以大位移(即大时间步长和/或大速度)在粗糙度上滑动,这些挑战只会增加。在这里,我们测试了C-IPC对极端情况的处理,在这样的情况下,如果我们从尖锐接触中具有极端拉伸,C-IPC的自适应应变限制刚度对于确保时间步长求解在数值上保持可行和高效尤为重要。

\textit{桌布把戏}。经典的桌布技巧试图从桌子设置下拉出一块布,目的是保持餐具直立,理想情况下大部分都保持在原位。使这个技巧起作用的关键是以足够的速度拉动,以便布料的滑动加速度可以克服摩擦力,而设置5上的拉力很小或没有拉力。因此,模拟这种效应需要将弹性和应变限制与摩擦和接触的精确分辨率紧密耦合,以及对锋利边缘滑动接触的可靠处理。这是正确模拟摩擦力、拉力和拉力以及重型餐具法向力之间的平衡所必需的。在图 1 中,我们为 C-IPC 设置了一个表格,其中包含重型和刚性餐具(FEM)。我们模拟拉动应变限制(1.0608 上限)桌布,时间步长为 $h = 0.01s$,速度增加。正如我们预期的那样,当我们从较低的拉动速度开始时,物体不会跟随布料。随着我们提高速度,我们得到的结果就会减少,最后,以 4m/s 的速度拉动,只需稍微移动一下,整个设置保持直立并放在桌子上(我们还注意到,高速拉动的应变限制布开始形成详细的折叠)。最后,通过 8m/s 的高速拉动,桌布平稳拉出,几乎完全不改变设置。在这里,对于后两次成功的拉动,我们观察到来自高速拉动布料的移动边界条件的极端力以及与较重餐具接触时施加的巨大摩擦力。

\textit{扭曲的圆柱体}。最后,我们测试了极端和增加应力下的厚度建模、接触分辨率和屈曲。我们开始用一个1m宽的布筒(0.25m半径),由具有88K节点的壳建模,厚度偏移为1.5mm,d为1mm,时间在$h = 0.04s$。然后,我们同时扭转圆柱体,并分别以 $72^{\circ}/s$ 和 5mm/s 的速度将两侧缓慢移动,以引入起皱、折叠和最终的紧屈曲。为了让这种脚本化的折磨持续整个38秒的序列,我们不应用应变限制 - 因此C-IPC必须在这里解决极高的拉伸力。同样,为了澄清屈曲行为,我们不施加重力(避免下垂)。在模拟的第一秒,我们立即获得有趣的全局折叠(请参阅我们的补充文档和视频)。不久之后,一个厚厚的中央圆筒的缠绕布形成。该圆柱体结构的体积由 C-IPC 的有限厚度偏移支持(图 2 左上角)。为了进行对比,请考虑图2右上角在同一时间步捕获的帧,但现在在没有C-IPC偏移的情况下进行模拟。在这里,我们清楚地看到,如果没有C-IPC的厚度偏移(它也不能捕获材料后来的屈曲行为 - 见下文),就无法形成正确的几何形状,这进一步阐明了一致厚度建模对同维模型的重要性。接下来,随着我们继续我们的 C-IPC 仿真,偏移 32.96 秒的进一步扭曲,我们的偏置加厚布继续支持复杂的行为,包括图 2 底部的最终屈曲几何形状。


\section{总结}
C-IPC 将增量电位接触框架从体积变形扩展到具有可控厚度、完全耦合应变限制、精确摩擦行为和在所有时间步保持非相交的共维和混合维结构。当然,与原始IPC方法一样,为了提供这些非相交保证,IPC需要在仿真开始时具有初始的非互穿几何形状。目前,例如对于服装,这通常可以通过分期轻松实现。然而,放宽这些约束,从开始纠结状态实现保证,是未来探索的重要且令人兴奋的方向。

正如我们的压力测试所证明的那样,C-IPC始终如一地收敛到最佳解决方案,即使受到挑战性边界条件的影响,这些条件会突破应变极限和接触屏障。然而,C-IPC不能在施加不可行的(例如,相交或违反应变极限)轨迹的边界条件下进行。在这里,检测和/或找到IPC通过不可行的条件的方法将是未来具有挑战性的扩展。对于摩擦,C-IPC直接利用IPC最大耗散的平滑半隐式离散化,并可选择滞后迭代。

正如在最初的IPC工作中所验证的那样,我们确认C-IPC准确地捕获了具有收敛滞后的粘滑阈值(请参阅我们的布滑动基准)。否则,对于C-IPC仿真,我们应用单个滞后迭代,并发现它足以模拟复杂的摩擦行为(例如桌布拉力和旋转球体)和详细的服装窗帘。然而,与最初的IPC工作(以及以前的非平滑方法)一样,我们也注意到,不能保证收敛到完全隐含的摩擦关系的准确满足。实现这一点的算法仍然是仿真面临的一个开放且重要的挑战。

除了在布料、头发和许多其他同维仿真任务中实现新的和改进的应用外,我们还很高兴能够进一步扩展C-IPC在颗粒流动模拟中的岩土力学应用,特别是对于迄今为止已证明难以解决的复杂颗粒形状。

在这里,我们首先专注于为所有材料和条件提供可靠的模拟器。如之前的工作所述,在许多应用中,最大的开销是手动调整碰撞参数以使场景正常工作所需的许多费力的模拟通道,然后通常需要更多耗时的步骤来校正输出。因此,可靠性导致整体输出速度的提高。然而,如上所述,尽管C-IPC的性能在提供可比的精度时与最先进的布料代码相比更具竞争力,但要提高C-IPC性能,还可以做更多的事情。我们期待更好地利用并行架构并改进底层 IPC 求解器方法。同样,在这里,我们正在调查并使用隐式方法进行广泛测试。另一个富有成效的探索方向是明确的时间步进。考虑C-IPC和ACM的相对行为,以便在提供非交集保证的最新隐式和显式方法之间进行并排比较,这将是令人兴奋的。6 同样,我们期待利用我们方法的可区分性。在这里,我们预计C-IPC在输入扫描方面的可靠性能及其可区分性的综合优势应该共同支持服装和复杂结构的许多设计和培训任务,如Macklin等人和Geilinger等人所证明的那样。最后,在这里,我们专注于仅提供具有上限的应变限制,虽然我们还没有看到需要下限的情况,但这些情况很容易应用,因此对于未来的探索应该很有趣。

虽然事实证明对C-IPC至关重要,但我们也表明,ACCD是预先存在的复杂且通常敏感的CCD模块的极其简单的替代品。在初步测试中,我们发现在C-IPC框架之外的应用中,ACCD也可以提供显着的加速。我们期待ACCD作为一种极其易于实现和高效的替代CCD算法,用于仿真和几何处理任务的进一步应用。

总之,与弹性模型和时间步长无关,C-IPC保证了开箱即用的无相交模拟,严格满足应变极限(确认低至0.1\%),并准确捕获几何有意义的厚度。这部分是通过一种新的、简单的、加法CCD(ACCD)方法实现的,该方法具有高度稳定和准确的输出,适用于具有挑战性的首次撞击任务。

(参考文献、相关图表见原文)

\end{document}
