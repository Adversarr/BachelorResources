
\section{引言}

薄的材料,如布,头发和沙子无处不在。长期以来,对其进行动力学模拟仍然是计算机建模和动画中的一项关键任务。每个这样的固体通常最好通过相应的还原共维公式建模,分别由壳、棒和粒子组成。这样做可以在模拟自由度(DOF)更少的情况下提高效率。同样,它减轻了\textit{剪切锁定}带来的数值条件和严重的收敛挑战,否则如果模拟薄材料体积,我们会遇到这些挑战。但是,虽然我们可以通过同维模型很好地捕捉单个薄材料的弹性行为,但在实际尺寸网格上准确、一致地模拟它们的集体和接触行为,使用真实世界的材料和条件仍然具有挑战性。我们在此列举这些挑战。

首先,模拟应始终保持\textit{无相交}。即使允许同维几何体的轻微相交的时间步也可能会产生模拟无法恢复的问题。

其次,我们需要能量守恒、可控和准确的应变限制。虽然共维离散化减轻了剪切锁定,但它们仍然可能遭受严重的\textit{膜锁定}。例如,壳模型在布料的弯曲模式下继续遭受严重的数值刚度,除非中表面网格应用极高的(通常是无法实现的的尺寸)分辨率。。在这里,高阶FEM方法也可以帮助进一步减少锁定伪影,但它们增加了令人望而却步的计算成本。应变限制允许使用较软的布料材料,并结合严格的限制,以避免这些伪影,同时仍能恢复适当的布料响应。但是,如果应变限制没有准确且一致地应用,并且小心地与弹性动力学耦合,则通常会产生比修复更多的数值伪影。

第三,必须在接触中捕获几何厚度。虽然薄结构通常具有较小的相对厚度,其弹性行为可以通过共维自由度间接捕获,但正确和直接地模拟它们的接触厚度对于捕获准确的几何形状和体积行为至关重要。例如,考虑一副纸牌。单独考虑时,每张卡片的厚度几乎难以察觉。但是,堆叠时,它们的组合厚度大而清晰,不容忽视。虽然接触加工策略通常会引入厚度参数,但它们通常用作非物理故障保护,以减少碰撞处理的不准确性。正如Li等人所分析的那样,这些厚度无法通过统一的方法的被实现,而且必须根据场景和示例(例如,基于碰撞速度)进行更改,以避免模拟失败。

第四,所有共维(包括体积)都应在一个通用框架中无差别地,也没有特殊的处理地无缝模拟。精确的摩擦接触力应直接耦合所有几何类型,无论接触有多紧密或建模厚度有多薄。

第五,具有小但\textit{有限}厚度的共维模型之间的接触建模对所有碰撞检测算法提出了挑战。最关键的是,我们看到所有可用的CCD方法和代码都存在不可接受的错误或效率很低。无论是标准浮点寻根的CCD方法,还是精确CCD方法的最新发展,我们都看到了这一点。这些问题可归因于所提出的具有挑战性的潜在寻根问题众所周知的数值敏感性。在这里,最具体地说,我们看到IPC依赖于所有CCD计算正确,作为无交的保证。足够大的保守边界可以帮助解释这些不准确性,确保非交集,但它们这样做是以算法效率为代价的,因此在涉及共维自由度时甚至会完全停滞收敛。

本文提出了一种方法。据我们所知,该方法是第一个通过一致、可靠的摩擦接触动力学解决方案,来直接解决上述所有六个挑战的方法。我们的方法保证严格满足完全耦合的应变限制,无交集轨迹和可控的几何厚度分辨率(在薄材料尺度下),不受时间步长和接触条件严苛程度的影响。

\subsection{主要贡献}

具体来说,为了应对这些挑战,我们扩展了增量接触势能模型(IPC)用于含接触的弹性体动力学,以建模并求解由具有三个关键部分的任意共维自由度组合组成的系统。

\begin{itemize}
\item[\textbf{本构应变限制}]我们引入了一个$C^2$的罚函数模型,该模型直接将应变限制作为能量势,同时保持静止状态。这为布料提供了能量一致的应变极限模型(各向同性和各向异性),首次能够严格满足所有迭代和所有时间步长的应变极限不等式(验证低至 0.1\%),同时通过最小化增量电位完全直接耦合到弹性动力学和摩擦接触。因此,如第6.3节所示,我们避免了传统应变限制方法中由力分裂误差产生的伪影。
% TODO: 6.3节
  \item [IPC 厚度模型] 为了捕获接触的同维域的几何厚度,我们扩展了IPC模型以直接执行距离偏移。我们的处理严格保证壳(分别是棒和粒子)的表面(分别为中线和点)不会比应用的厚度值更近,即使这些厚度变得非常小(见第5节)。这使我们能够考虑同维结构接触行为中的厚度,从而捕获具有挑战性的接触效应。
  \item [增量CCD] 为了提供CCD所需的严格精度来解析具有厚度的共维模型,我们利用保守步场开发了一种新的,高效且易于实现的CCD(ACCD)方法。ACCD反复累加一个收敛于碰撞时间的下界,并且对于所需的极具挑战性的求解都是稳定的。虽然我们最直接关注的是ACCD在C-IPC框架内的应用,但正如我们在以下各节中所示,与之前的所有CCD算法(精确和浮点)相比,ACCD非常简单且易于实现,具有改进的性能,鲁棒性和保证,因此适合于所有使用CCD模块的应用中轻松更换。
\end{itemize}

这些共同构成了同维IPC(C-IPC)的核心。C-IPC 能够对所有共维进行统一仿真,包括弹性体积体、壳、棒和粒子,所有这些都通过精确求解的接触和摩擦耦合在一起。除了这些核心贡献外,我们还在许多新的和已存在的基准中测试、比较和分析C-IPC。我们确认C-IPC提供严格可控的几何厚度行为,并保证所有时间步的可行性。最后,C-IPC保持无交集,并严格满足所有计算轨迹上的所有设定应变限值,同样在我们广泛的基准测试中得到了验证。






