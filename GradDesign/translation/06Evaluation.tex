\section{性能测试}

我们在C++中实现了我们的方法,应用了 CHOLMOD [Chen 等人,2008 年],使用英特尔 MKL LAPACK 和 BLAS 编译用于线性求解,用特征编译用于剩余的线性代数例程 [Guennebaud 等人,2010]。C-IPC的必要导数以及我们应用于效率和数值鲁棒性的代数简化在我们的补充文档中有详细说明。为了实现未来的应用、开发和测试,我们将把C-IPC的实现作为一个开源项目发布。我们所有的实验和评估都是在英特尔 16 核 i9-9900K CPU 3.6GHz × 8(32GB 内存)、英特尔酷睿 i7-8700K CPU @ 3.7GHz × 6(64GB 内存)或英特尔酷睿 i7-9700K CPU @ 3.6GHz × 4(32GB 内存)上执行的,详见下文实验。

\textit{实验}:下面我们从一项研究开始,说明和分析标准布料和网眼的膜锁定行为(第 6.1 节)。然后,我们在6.2节中展示了C-IPC在完全耦合所有物理力的同时严格满足应变极限的能力。据我们所知,C-IPC是第一种既能严格满足应变极限,又能支持极限、弹性和接触力完全耦合的方法。为了分析应变极限满足和完全耦合的影响,我们接下来考虑与先前方法的比较。如第2节所述,应变限制中的现有方法会产生包括锁定、抖动和互穿在内的伪影。这些问题有两个来源:1)应用的分裂模型不准确,2)无法满足计算中的应变极限。在这里,我们首次分别分析了与求解器精度误差无关的拆分模型(第 6.3 节)所产生的问题。然后在第6.4节中,我们考虑了由于分裂误差和方法无法执行要求的应变限值而导致的伪影和不准确性(由最先进的布料代码产生)。在6.5和6.6节中我们分析了C-IPC的厚度建模,与现有的布料模拟规范和以前的CCD评估方法进行了比较。最后,在评估了C-IPC的所有组件后,我们评估了C-IPC在服装模拟中的应用(第6.7节)及其对先前压力测试布料基准的解析(第6.8节)。然后,我们在具有一致厚度建模的任意余维全耦合系统的仿真中演示了C-IPC(第6.9节),并最终考虑了其在一组旨在锻炼鲁棒性和准确性的新布料模拟应力测试中的性能(第6.10节)。所有实验设置和统计数据都列在图 24 中。我们报告系统中涉及的节点总数,包括运动学(相等约束)的节点。所有示例都直接应用 IPC 的平滑半隐式摩擦(1 次滞后迭代)和默认牛顿公差(如果未另行提及)。

\subsection{布料材质}

在这里,我们首先说明并检查膜锁定对现实世界材料的影响,以及应变限制减轻它们的能力。在Penava等人[2014]中,测量并验证了一系列布料的密度,厚度和方向依赖性的膜应力 - 应变曲线。然而,在具有标准分辨率网格的仿真中直接应用这些真实世界的布料参数会产生严重的膜锁定。这是不可避免的,除非使用非常高且经常不切实际的大网格尺寸。在这里,我们检查这种锁定行为,首先使用我们的IPC模型进行演示,没有应变限制。我们还注意到,这种锁定行为与算法无关,因此,例如,也很容易在其他代码中演示,例如 ARCSim [Narain 等人,2012 年],以及它们在现实世界中捕获的材料参数 [Wang 等人,2011 年];参见图 15d。

我们首先考虑一个简单的1m×1m非结构化网格在固定球体和地平面上掉落的行为(所有摩擦力为$\mu = 0.4$)。我们应用Penava等人[2014]测量的棉布密度($472.6kg/m^3$)和厚度(0.318mm),然后考虑改变膜刚度,同时保持弯曲杨氏模量为0.8MPa,两个膜和弯曲的测量值为经纱方向的测量值为0.243。特别是Penava等人[2014],发现膜杨氏模量范围从0.8MPa到32.6MPa,用于不同的平面内方向。

对于我们的示例,我们发现,即使应用各向同性膜模型,具有最小的确定膜刚度(0.8MPa),我们也观察到严重的锁定伪影(参见图11a),其中弯曲是人为硬化的。如果我们接下来将这个最小的测量膜刚度降低0.1×,那么我们会看到人工弯曲伪影在很大程度上被消除,但我们仍然获得由主导膜能量强迫的尖锐折痕伪影(参见图11b)。接下来,如果我们尝试更小的0.01×膜刚度缩放,可观察到的膜锁定效应现在已经消失,但正如预期的那样,得到的材料被拉伸得太过长,因此甚至没有接近所需的材料行为(图11c)。这个简单的例子很好地演示了在模拟刚性真实世界布料时膜锁定的挑战。然后我们注意到,这些伪影只会在更具挑战性的模拟中加剧,例如移动边界条件和紧密接触。

接下来,我们使用C-IPC的各向同性模型应用严格的应变限制,以将应变限制在Penava等人[2014]测量的弹性范围内。棉花在各个方向上的最宽范围允许拉伸系数高达 6.08\%。在这里,应用此边界,即使膜刚度缩放到0.01×,我们现在重新获得没有膜锁定的模拟,并且由于仅限于测量的应变极限,也没有不自然的拉伸伪影(图11d)。

最后,需要重申的是,膜锁定确实是一种与分辨率相关的伪影,其影响会随着网格尺寸的增加而减小。例如,在 1× 节点的网格下,覆盖在球体上的布的锁定伪影在 85K 节点时弯曲刚度显着降低,而在 246K 节点处则难以察觉。但是,这种改进在很大程度上取决于场景参数。例如,在完全相同的场景中,相同的246K网格表现出明显的锁定伪像,如果我们只是将弯曲刚度降低到以下方面。

0.01×与图6底行应变极限模拟中的材料相匹配。请参阅我们的补充文件,了解这些案例的详细信息和说明。

\subsection{精确应变限制}

以上我们已经证明了应用应变限制在模拟布料材料中的重要性和影响。在这里,我们研究了C-IPC在我们的各向同性和各向异性模型中准确执行严格应变极限的能力。

\textit{在降低应变极限的范围内保持精度}:为了确认 C-IPC 本构应变极限的准确性、可控性和鲁棒性,我们测试了越来越严格的 1\% 和 0.1\% 应变极限(远低于标准布料中的测量极限)。我们将它们应用于上一节中考虑的球体悬垂示例以及使用两角钉布进行的简单膜锁定测试[Chen等人,2019;金等人,2017]。在这里,应用 C-IPC,参见图 12,我们观察到稳定的仿真输出,所有三角形上的拉伸都满足其规定的限制约束。仔细检查图 12 中的悬垂球体测试也证实两者都不存在人为弯曲刚度。相比之下,将这些结果与图11a中的人工硬化进行比较,尽管直接应用了测量的膜刚度,但最大拉伸超过4.5\%。

在这里,当我们单方面应用应变限制时,这与Jin等人[2017]的论点一致,即单边应变限制可以比双边执法更好地避免膜锁定 - 至少在一定程度上。如果限制太严格,即使是单边执法也不是完美的灵丹妙药。举一个极端的例子,我们可以将应变极限提高到一个非常小且基本上不切实际(0.1\%)的应变极限。这提供了一个极端的压力测试,我们确认C-IPC确实在所有时间步长中保持了这些具有挑战性的限制。然而,正如预期的那样,我们最终可以在图12b中观察到一些轻微但明显的锐利边缘折痕伪像,在图12d中观察到类似可见的锁定问题。尽管这种极端极限在实践中不太可能发生,而且布料通常也不会遇到,但该实验确实强调了一个重要的观点:如果大多数单侧应变极限约束是有效的,那么它们的行为与双边情况相似。因此,当大多数应变极限约束处于绑定活动约束数时,可以再次接近自由度的数量,因此,如Chen等人[2019]所述,可以再次遇到锁定。在这里,我们专注于制定一个可控且鲁棒的应变极限本构模型。我们希望这将通过探索应变约束的交替离散化来进一步增加无锁定配置的范围,如Chen等人[2019]。但是,我们注意到,当此处提出的底层方法能够准确地保证耦合的约束满足时,可以最好地启用此功能。锐利边缘折痕伪像,在图12d中观察到类似可见的锁定问题。虽然如此极端的极限

\textit{各向异性应变限制}:对于各向异性,情况类似。在图 13 中,我们演示了 C-IPC 的各向异性模型,包括球体悬垂和双角悬挂测试。我们再次观察到,应用Clyde测量的刚度值会产生清晰的膜锁定伪影(见图13a)。接下来在图13b和c中,我们分别将膜刚度缩小0.1×和0.01×。在整个过程中,我们确认C-IPC执行了克莱德测量的应变极限,并再次获得了预期的改进结果,既没有锁定伪影,也没有非物理拉伸,刚度为测量刚度的0.01×。同样,在图13d中(同样是0.01×报告的刚度),我们应用C-IPC来保持测量的应变极限,使用各向异性模型获得无伪影的皱纹。

\subsection{对比算符分裂模型}

现有的应变限制方法引入了来自两个主要来源的误差。首先,在建模级别,他们将应变极限与弹性分辨率分开。其次,无论采用何种分裂模型,用于求解它们的算法都是不准确的,并且通常无法满足甚至中等要求的应变极限,正如我们将在下一节中看到的那样。然后,所有先前的方法都会引入来自这两个来源的错误,因此尚不清楚每个来源都产生了哪些问题。在这里,我们首先分别分析拆分模型产生的问题,通过精确地解决拆分的每一步。这使我们能够证明,无论如何准确地执行极限约束,拆分本身都会引入不可避免的错误。然后,在6.4节的下一步中,我们将检查求解精度,并看到应变极限求解本身中的误差会产生不一致且通常无法控制的仿真结果。

为了检查分裂误差,我们首先应用标准的应变限制,分裂策略,因此将每个时间步长求解分为两个连续步骤。第一个求解一个预测器步骤,该步骤包括整个系统的所有力,除了应变限制,以获得中间配置$\hat x$。接下来,第二步预测预测变量以满足应变限制。为此,我们最小化本构应变极限能量,总和为从最终时间步长解到$\hat x$的质量加权$L^2$距离。

在最简单的情况下,我们可以考虑在没有接触力的情况下这种分裂的影响。我们从固定和拉动的方布开始。我们固定它的两个顶角,并施加重物将其两个底角向下拉;参见图 14a 和 b。相比之下,在没有应变限制的情况下,布被拉伸超过10×,而在我们的本构应变极限下,应变被限制在测量的弹性范围内,在这个完全耦合的解决方案中,我们看到预期的垂直对齐的皱纹在两个底部拉角处拉伸;参见图 14a。另一方面,在图14b中,我们看到弹性求解的分裂应变限制在皱纹在非物理方向对齐的两个底角产生明显的偏置伪影。这些误差是由应变限制力的解耦引起的,随着时间步长的增加而变得越来越严重(这里我们用$h = 0.04s$)。

接下来,为了考虑接触的分割误差,我们再次考虑球体上的布料示例。在这里,我们应用新胡克膜弹性来帮助分裂方法避免可能的三角形简并。请注意,即使在这里,我们的完全耦合的C-IPC模型实际上也不需要新胡克弹性(因为三角形退化是通过邻居之间的点三角形约束来防止的)。然而,为了保持一致的比较,我们对两者应用了Neo-Hookean模型。在图14d中,我们看到接触的两步分裂现在受到严重的压缩伪影的影响,这再次来自第一步的弹性动力学解析,然后在第二步中施加应变限制和接触力。为了进行比较,请考虑图14c中C-IPC相应的全耦合应变极限解决方案。反过来,分割解中的误差表明,应变的额外下限(例如,在 ARCSim 中应用的应变)可能有助于避免这些压缩误差。如果我们另外添加一个应变极限下限(此处为 0.7),那么我们确实会发现得到的分裂解决方案现在没有严重的压缩伪影。然而,现在,由于膜和弯曲的错误,分裂溶液中的布仍然不自然地平放在地板上;参见图 14e。

当然,与所有时间步长拆分方法一样,可以通过应用越来越小的时间步长来减少拆分误差。例如,在这里,我们发现分割模型和完全耦合求解之间的视觉上明显的误差在$h = 0.004s$时消失了。但是,正如预期的那样,计算时间的大幅且通常不可接受的增加抹去了拆分的任何预期收益(我们将在下一节与现有布料代码的比较中再次看到此主题)。同样,时间步长的必要减小随示例和场景而变化,因此似乎不可能实现具有拆分功能的稳健、可控的时间步长。

\subsection{对比ARCSim的应变限制求解器}

上面我们分析了应变限制的时间步长分裂产生的建模误差。为此,我们比较了每个阶段在通用基准代码中始终如一地求解的模型。接下来,我们考虑拆分误差与不准确的约束求解相结合在现有布料仿真代码中产生的联合影响。为此,我们与ARCSim进行了比较,据我们所知,这是目前最强大的外壳模拟器,可以支持真实世界布料材料参数的应变极限。有关 ARCSim 和 C-IPC 实现的应变极限和时序的统计数据,请参见图 16。

我们再次从考虑更简单、无接触的情况开始。为此,我们考虑一个更简单的布挂示例:通过放下没有加重底角的固定布。我们应用 ARCSim 的默认棉材料和默认应变极限设置,这些设置限制在$ [0.9, 1.1] $范围内。在这里,即使在如此广泛的允许应变和全膜刚度下,ARCSim结果在每个时间步都明显违反了极限界限。为了直观地参考图15a,我们说明了满足应变极限界限的平衡下的完全耦合C-IPC解决方案。与图15b中平衡时的ARCSim输出进行比较,我们很容易看到固定角附近过度拉伸形成较大弧线的伪影。这里两帧都是在 4s 拍摄的,并且时间步长为 $h = 0.04s$。

当我们将 ARCSim 的时间步长减小到 $h = 0.01s$ 时,这些误差仍然很大,直到我们达到 $h = 0.001s$ 的步长时,应变极限才基本(尽管仍然不完全)得到满足。然而,要做到这一点,ARCSim需要87分钟来模拟这个简单的4s仿真序列,与C-IPC的完全耦合、满足应变极限的解决方案(在h = 0.04s时步进)相比,速度减慢了10×以上。然而,除了必要的减速之外,每个场景和材质所需的步长减小变化,因此总是不清楚,如果没有每个场景进行许多耗时的模拟测试,如何确定必要的时间步长减少,以确保 ARCSim 结果满足规定的应变限值。相比之下,C-IPC再次在时间步长、材料和场景设置方面保持应变极限约束满足。为了进一步研究ARSim的约束误差,我们还观察到ARCSim中使用的增强拉格朗日应变限制求解器应用了设置为100的硬编码最大迭代上限。通过实验增加这个极限(否则保持ARCSim代码不变),我们确认即使将这个上限增加一个数量级,应变极限仍然没有达到满足。

接下来,我们为 ARCSim 比较添加联系人,并考虑 8K 节点球体悬垂测试。在这里,我们再次观察到,应用应变限制来减少膜锁定是行不通的,相反,随着膜硬度的降低,应变极限满意度会恶化,伪影实际上会增加。

具体来说,我们首先应用 ARCSim 的默认棉质材料和与上述相同的默认应变极限设置。对于这种刚性材料,我们观察到预期的膜锁定问题;参见图 15d。然而,我们确实注意到,对于这种刚性膜,除了几个时间步长之外,应变极限约束肯定得到了很好的满足。

接下来,作为标准,我们降低 ARCSim 中的膜刚度,希望减轻锁定,并期望应变限制将补偿。如果我们将拉伸刚度降低 10×,锁定伪影确实更少;但是,ARCSim的应变限制不能为刚度降低提供预期的补偿。相反,在整个仿真过程中,违反应变极限的三角形明显更多(见图15e),导致拉伸伪影。如果我们进一步将刚度降低到默认棉布值的0.01×(在本例中通常是去除所有可见锁定伪影所必需的;例如,与我们的可比刚度进行比较,C-IPC结果如图11d和15c)ARCSim在图15f中的结果显示更明显地违反了应变极限,因此遭受了与上一节中分裂模型分析相同的极端压缩问题;有关比较,请参见图 14e。此外,我们现在还观察到 ARCSim 进一步的三向分裂引起的抖动,该分裂进一步将应变极限求解和接触力分成两个独立的顺序投影步骤。我们注意到,最后一个问题在我们的分析中并不陌生,Narain 等人对此进行了讨论。

\subsection{厚度建模}

接下来,我们研究接触中小而有限的厚度建模。为了说明C-IPC模型与现有最先进的外壳模拟器之间的差异,我们检查了一个简单但具有挑战性的布料堆叠基准。我们构建了一个场景,其中堆叠了十个 8K 节点的方形布面板,这些面板以垂直间距和旋转方向对齐,同时放在固定的方形板上。有关初始设置,请参见图 17a。这应该形成一个布堆,当我们改变材料厚度时,整体桩高和体积特性也应相应变化。

\textbf{ARCSim} [Narain et al. 2012] 计算了 3 个碰撞响应,应用 Harmon 等人 [2008] 的非刚性撞击区和应用于模型布厚度和稳定碰撞处理约束的排斥厚度。ARCSim的厚度参数默认值设置为5mm,并显示为可由用户更改的配置。然后在代码中将此参数与直接应用于碰撞投影、分析和接触力计算的其他厚度参数相关联。从默认的 5mm 厚度设置开始,以越来越小时的时间步长进行测试,低至 h = 0.001s,ARCSim 始终报告碰撞处理失败,并在仿真结果中表现出严重的伪影;参见图 17b。接下来,应用10mm厚度的ARCSim报告在h = 0.001s时成功解析接触,但仍会在几何和动力学中产生较大的伪影;参见图 17c。我们没有将 ARCSim 进一步推到小于 h = 0.001s 的时间步长,因为这已经使 ARCSim 的计算速度过慢;例如,与C-IPC相比。

\textbf{ARGUS} Li等人更新了 ARCSim,改进了用于壳体仿真的接触和摩擦处理。与 ARCSim 一样,它也应用并向用户公开相同的厚度参数。使用默认求解器参数,在h = 0.001s时,ARGUS能够模拟默认5mm材料厚度(图17e)和较厚的10mm材料(图17f)的布料堆叠。在这里,我们还观察到桩中的一些高度差被捕获。然而,当我们减少厚度,使更薄、更标准的服装材料厚度减少时,例如将厚度设置为1mm,ARGUS会产生严重的伪影,如图17d所示。与ARCSim一样,对于所有三种厚度设置,ARGUS的计时仍然比C-IPC慢得多。

总之,我们观察到,最先进的方法需要很小且通常不切实际的时间步长,以避免随着建模几何厚度的减小而失败。随着厚度的减小和/或碰撞速度的增加,避免故障所需的时间步长会减少,从而相应地增加仿真成本,并且不清楚每个场景和厚度成功所需的时间步长。

\textbf{C-IPC} 对于相同的厚度基准,只需将$\hat d$(回想一下我们开始施加接触力的距离)设置为有效厚度值1mm,5mm和10mm,即可始终对不同的厚度进行建模。在图18的第一行中,C-IPC连续模拟所有不同的材料厚度和相应的布料堆叠高度。然后,C-IPC的模型进一步提供了法线方向上厚度的本构行为。图 18 的中间行和底部行对此进行了演示,然后我们将一个弹性球放在堆栈上。在中间一行中,我们显示了每个堆栈碰撞的最大压缩,观察随着厚度的增加而增加的压痕。另请注意,皱纹仅在最厚的 10 毫米外壳中形成。在底行中,我们显示了碰撞后每次模拟的最终平衡帧,说明了由体积有限元球模型加权的壳堆的不同静止高度。这个例子还说明了C-IPC对不同弹性模型的自然耦合,我们将在第6.9节中详细探讨。这些示例还有助于说明 C-IPC 的计算成本如何随阈值 $\hat d$ 而变化;参见图 24。一般来说,我们看到在较大的阈值(例如$\hat d$ = 10mm)下,C-IPC每步处理更多的触点对,使得仿真比在较小值下更昂贵,每次求解的触点更少,例如$\hat d$在1mm和5mm处。同时将d设置为小几个数量级的值,例如1 $\mu m$,由于势垒函数的清晰度增加,收敛速度较慢,因此计算时间更长。

\subsection{CCD 评测}

对于最后一部分的比较,仅通过d模拟C-IPC中的厚度效应就足够了,因此纯粹是一种弹性行为。更一般地说,如第5节所述,随着厚度变小,接触更紧密,需要同时使用$d$和偏移$s$的C-IPC非弹性厚度模型。

为了模拟共维对象,这两种机制结合在一起。在这里,较大的 $\hat d$ 引入了增加的弹性海绵,而非零 $s$ 即使在极端压缩下也能保证最小厚度;例如,参见图2。对于厚度一致建模很重要的场景;见例如。在图2、3、10、21、22和23中,我们再将s设置为所报告的物体的材料厚度值或稍小,然后在$t$附近设置$d$,以控制所需的弹性响应程度。

如第5.3节所述,仿真这些示例会产生更多的简并接触对,并且需要更高的精度来捕获厚度偏移,因此对CCD精度提出了很高的要求。在这里,IPC的保守 CCD 策略已不再足够。使用标准浮点CCD例程可以并且将简单地返回许多查询的TOI,因此会错误地完全停止IPC优化进度。同样,在大多数情况下,具有更高报告精度的替代CCD方法同样失败。

这激励了我们新的ACCD方法,如下图所示,该方法仍然稳健而准确 - 即使在我们所有可用的CCD替代品都失败的最复杂的例子中,也能在IPC优化方面取得稳定,向前的进展。与此同时,ACCD继续在替代CCD方法能够成功的较简单例子上提供可比且通常更快的时序。

为了与ACCD进行比较,我们测试了一系列CCD方法:最小分离(MinSep)CCD 、根奇偶校验、BSC、T-BSC和 IPC 的保守浮点 CCD,一组十个具有具有挑战性的同维碰撞的基准示例。基准测试中的前五个示例具有厚度偏移,其余五个示例没有;参见图 19。

我们能够通过在与IPC相同的保守CCD策略中将其用作基础求解器来直接测试MinSep。此处的最小分离设置为$s \times d^{cur}$,当前比例因子 $s$ 和距离 $d^{cur}$ 。如果返回的 TOI 小于 $10^{−6}$,则再次执行不缩放($s = 0$)的查询,以尝试使查询更容易。为了测试根奇偶校验、BSC 和 T-BSC 方法,它们只决定一个区间是“有冲突”还是“没有冲突”而不计算 TOI,我们对每个查询应用它们而不回溯,直到找到没有冲突的步长。找到后,我们返回步长的保守缩放 $1 − s$。由于回溯不能直接将这些方法扩展到计算首先将接触对的距离带到ξ的时间或步长,因此我们无法在具有厚度偏移的示例上测试这三种方法;因此,我们将对它们的测试限制在图 19 中的最后五个测试示例中)。

我们在图 19 中总结了我们的比较。IPC使用浮点CCD的保守策略在更简单的场景中效果很好,尽管时序比我们的ACCD方法稍慢,但在更复杂的场景中却失败了。我们还注意到,虽然保守的浮点CCD通过查询额外的点-边和点-点对来处理平行边-边等退化情况,但ACCD只需要查询一般对(例如,仅表面场景的点三角形和边-边对)。另一方面,MinSep 总是为所有十个基准示例返回微小的值,有时甚至是 0,因此没有任何进展。根奇偶校验在它可以应用于的大多数基准示例(没有厚度偏移的基准示例)上表现良好,效率很高,但不幸的是,它完全错过了一些碰撞,导致针床和服装等示例中不可接受的相互渗透。对于 BSC,我们发现它报告了所有基准示例的运行时错误,可以使用内部消息“拐点在 BSC 中未完全处理”来应用,而 T-BSC 返回服装示例的微小值,并且无法在没有碰撞的情况下确定步长(我们确认两种 BSC 变体在简单情况下都运行良好),因此在它可以应用于的其他五个基准示例中陷入行搜索。相比之下,我们看到ACCD使所有示例能够完成收敛,而对于替代CCD例程能够完成的示例,它提供了最快的时间。最后,作为对C-IPC框架之外的ACCD应用的初步调查,我们应用ACCD代替阿格斯默认的CCD例程,并使用1mm和5mm布堆叠示例进行测试。虽然正如预期的那样,仿真结果几乎相同(没有视觉差异),但CCD成本降低了约50\%。

\subsection{服装模拟}

接下来,我们应用C-IPC来模拟服装的悬垂和动态。

阿格斯比较。我们首先考虑C-IPC在ARGUS提出的一对测试舞蹈序列。这两个示例都使用了Li等人中测试的最高分辨率输入网格。在这里,杵状指练习接触处理更广泛,动作和过渡比蔓藤花纹更快。为了获得可比较的结果,我们在同一台机器上运行C-IPC并重新运行ARGUS,ARGUS的网格分辨率固定。在这里,我们在有和没有应变限制的C-IPC的整个序列中观察到稳健、平滑的仿真结果。值得注意的是,即使增加了严格的非交叉执行、收敛、完全隐式的时间步长和解析所有接触模板,而不是像 ARGUS 处理中那样仅解决节点节点,我们仍然观察到 C-IPC 在蔓藤花纹和 Clubbing 上的加速比 ARGUS 结果高 3.7 倍和 1.5 倍。同样,如果我们也启用C-IPC的本构应变限制,加速比同样保持在2.7X和1.4X。有关比较时序,请参见图 20,有关结果,请参见我们的补充视频。

\textit{服装模拟挑战}:服装仿真中的两个主要挑战是解决复杂的多层设计,以及模拟快速移动的角色所穿的服装。为了测试C-IPC在这些情况下的鲁棒性,我们构建了两个新示例。由于其复杂的缝合图案,这些服装无法在ARGUS和ARCSim代码中模拟。

我们模拟了 FoldSketch 制作的黄色连衣裙(15.7K 节点),其中创建了难以解决的刀褶,其中布料折叠在自身上,然后在接触时缝合紧。我们以围绕 12.8K 节点人体模型的平面模式开始 C-IPC(图 5a)。然后,我们通过向前推进CIPC来悬垂,拼接力(弹簧)以大时间步长将设计缝合在一起。C-IPC在经过几个大的时间步长后获得垂坠服装的静态平衡,获得锋利的刀褶,同时保持正确的分层顺序,没有交集(图5b)。然后我们将刀褶通过伦巴舞的台阶,保留了带有刀褶的连衣裙的流畅运动,保持清脆并在整个过程中适当碰撞(图5c)。

然后,我们应用C-IPC来类似地初始化使用敏感时装设计的多层裙子图案(30.5K节点)[Umetani等人,2011]。这得到了图5d中的初始姿势静态悬垂。接下来,我们编写穿着分层服装的人体模型网格,通过快速移动的武术序列 4 以大动作。在图5的底行中,我们显示了生成的C-IPC仿真的三帧,其中包含快速踢腿序列的一部分。另请参阅我们的补充视频。C-IPC稳定地解决了这些快速运动,以h = 0.04s的大时间步长继续,同时捕获复杂的服装细节,在整个过程中执行应变限制和非交集(从而保持分层顺序)。最后,我们将相同的服装通过较慢的伦巴序列,并观察到成本相应降低,因为需要更少的迭代来解决具有更温和运动和相应减少接触次数的时间步长。

\subsection{布料模拟基准测试}
在本节中,我们确认了C-IPC在先前工作中具有挑战性的基准测试中的表现。

旋转球体上的布。在图6中,我们测试了Bridson等人[2002]设计的旋转球型示例,这些示例旨在进行摩擦,接触处理和紧密起皱。我们首先将一块 86K 节点方布($\mu$ = 0.01 和 1.0608 应变上限的非结构化网格)放到球体和地面上(均为 $\mu$ = 0.4)。当球体旋转时,摩擦力会抓住布料,当布料向内拉动时,会捕捉到细小的起皱,而不会造成锁定和拉伸伪影。我们确认,随着我们提高分辨率(至246K节点)并降低弯曲刚度(降低10×),更高频率的皱纹在褶皱中消退。

漏斗。为了测试分层、摩擦和与应变限制的紧密接触,我们扩展了 Tang 等人 [2018] 和 Harmon 等人 [2008] 的漏斗示例。在图 7 中,我们将三块布板($\mu$ = 0.4,每个 26K 节点,应变上限为 1.0608)放在漏斗上。在摩擦下,它们稳定地放在顶部,直到一个尖锐的、脚本化的碰撞物体(四节点四面体)迅速将面板向下推并通过漏斗。C-IPC求解序列中的所有时间步长以收敛,同时始终保持无交集,应变极限满足轨迹。

礁结。为了将极端压缩,接触和摩擦与大应力相结合,我们匹配了Harmon等人[2009]的礁结示例的初始设置。两条丝带最初交织在一起,然后拉伸形成图 4 中的结。在这里,我们通过模拟100K节点(原始节点的10×)来扩展原始测试的挑战,施加1.134的应变极限,摩擦$\mu$ = 0.02,并拉得更紧,将带拉伸到接近其应变极限。这捕获了轨迹的摩擦动力学(没有抖动),并在中心处产生极端应力,形成比最初显示的要小得多的结。

\subsection{壳体、棒、颗粒的耦合}

在本节中,我们将演示跨所有协域的C-IPC仿真。这包括体积有限元材料、壳体、棒材,甚至在需要时还包括颗粒。由于这些域通过IPC屏障相互作用,因此我们能够在同一仿真中无缝且直接地耦合所有这些基于网格的模型。这些测试还突出了我们一致且可控的厚度建模对这些薄材料相互作用的优势和适用性。

\textit{洗牌}。我们在复杂的交错洗牌过程中测试非常薄的壳体摩擦接触的准确性和精度。我们从 54 张硬壳扑克牌(63.5 毫米× 88.8 毫米)开始,IPC 偏移和d分别设置为 0.1 毫米和 0.2 毫米,以解析卡的厚度。在图 8a 中,我们将卡片分成两堆,并在它们的侧面应用移动边界条件来弯曲它们,为“精确”桥洗牌做准备。与人类表演的桥牌完成不同,卡片开始交错,在这里我们通过保持两个桥桩分开来增加挑战。然后,我们从下往上逐张卡片释放保持的边界条件,在左右侧桩之间交替(图 8b)。释放的牌迅速恢复平坦的休息形状并落在地上,形成一个新的、洗牌的、无交叉的堆叠堆。然后,我们用两个脚本碰撞边界(图 8c)对甲板进行部分平方,得到一个 15.4 毫米高的最终桩,并且与真实甲板(图 8d)。在这里,我们应用了0.5×的默认公差来捕捉复杂的动态。

\textit{毛}:通过自碰撞对头发进行建模是杆接触加工的严峻挑战 - 部分原因是它们的横截面极小。我们在两个具有挑战性的头发模拟测试中检查了C-IPC的行为。头发厚度很小,但其体积效应很重要,并且仍然难以捕捉。在这里,C-IPC通过对所有$\hat d = 0.1mm$的杆强制进行0.08mm的偏移来捕获头发厚度。在图3(a)-(e)中,我们首先用编织测试C-IPC。我们通过相互扭曲两簇头发的底部自由度来形成一个紧密的辫子。尽管承受了很大的应力,但产生的编织物仍然没有相交,并在整个过程中保持厚度偏移。我们还通过释放底部保持边界条件并让辫子松开而不会造成伪影或不稳定因素来在动力学中证实这一点。接下来,我们通过将一簇具有一个自由端的头发放在类似的垂直悬挂集群上来进一步测试头发自碰撞和摩擦[McAdams等人,2009]和图3(f)-(k)中的固定球体。C-IPC再次模拟了这个场景,并在整个过程中进行了收敛的大时间步长求解。

\textit{面条}:一大碗面条很美味,但由于大量的自由度和高接触次数,模拟起来也很困难。在图 10a 中,我们将一堆 625 根 40 厘米长的煮熟的面条放入直径 13.85 厘米的碗中(固定几何形状)。每根面条都是一根 200 段离散的杆,$\mu$ = 0.4,偏移和 $\hat d$  分别设置为 1mm 和 0.5mm,以模拟面条的 1.5mm 厚度。随着C-IPC的厚度处理,掉落的杆停止(它们的总体积应为4.42×10 −4 m 3),主要填充碗(图10b),其体积为6.96×10 −4 m 3。在平衡时 在图10c中,我们移除碗以显示最终交织,无相互渗透的配置。最后,在图 10d 中,我们删除了体积渲染,以突出显示这个复杂的非相交体积仅使用离散杆折线进行模拟。

洒。同样,我们可以只对边缘段做同样的事情。我们将一个由 25K “洒”组成的网格,每个“洒”由 6 毫米长的单边缘段形成到一个碗中(图 21a)。洒水的 2mm 厚度由 1.5mm 的 C-IPC 偏移和 0.5mm 的 d 建模。使偏移量比 d 大 3× 模型是其碰撞的强非弹性响应。与面条一样,我们可以估计洒水的总预期体积为 $5.76 \times 10^{−4} m^3$,在平衡时填充碗的 $6.88 \times 10^{−4} m^3$ 体积,没有相互渗透(图 21b 和 c)。同样在图 21d 中,我们删除了体积渲染以突出显示仅使用同维线模拟的几何图形段。对于面条和撒布,我们使贴合的圆柱形表面比$\xi + d$ 薄 0.2mm。

布上的球形颗粒。在极端情况下,C-IPC甚至可以通过将单个顶点视为具有严格厚度偏移的颗粒来模拟小球形颗粒。我们通过将一列 50K 粒子滴到带有固定角的布上来测试非相交粒子轨迹的分辨率(图 22)。即使在高速行驶和与布料下,碰撞也不会相交。静止颗粒捕获紧密的球形填料(参见放大图),确认偏移保留。一旦颗粒堆积在中心,我们就会释放一个布角,让它们流向地面。为了模拟每个粒子,我们计算质量$\rho= 1600kg/m^3$和$V = \frac{4}{3}πr^3 \approx 4.2×10 −9 m^3$,并设置厚度,C-IPC偏移为2mm,$\hat d$为1mm。以这种方式模拟粒子类似于离散元法(DEM)[Cundall and Strack 1979;Yue 等人,2018 年],其中粒子被“涂覆”有相互作用区(在我们的设置中为 $\xi < \hat d < \xi + \hat d$),用于根据区域交点的深度、面积或体积确定接触力 [Jiang 等人,2020]。DEM是地质力学中的关键工具,它通过忽略粒子变形来提供合理的简化,同时准确捕获动量和粒子间的接触力。在这里,C-IPC通过我们的屏障功能将接触力直接与距离联系起来,因此可以与可变形材料无缝耦合,就像这里的布料一样。与DEM不同,C-IPC还额外保证了由偏移定义的粒子体积的无相交轨迹。

全包。最后,我们测试将所有余维耦合在一起。我们将FE犰狳体积,贝布和颗粒柱放到由交错杆形成的网上(图23a),对所有域施加0.5mm的C-IPC偏移和1mm的$\hat d$。我们一次依次删除这些共域。首先,掉落的犰狳被缠住并夹在网的杆之间(图23b)。接下来,布落在犰狳上(图23c),最后柱子倒在布上,流入由犰狳和杆网支撑的两个布褶皱中(图23d,e和f(俯视图))。

\subsection{布料仿真的新测试}

最后,我们在一组三个新的、具有挑战性的布料仿真实例上提出并评估了C-IPC,这些仿真示例在模拟具有摩擦接触、应变限制和厚度的弹性动力学时需要极高的鲁棒性和准确性。据我们所知,C-IPC是实现这些场景和行为模拟的第一种方法。

\textit{在同尺寸针上拉布}。众所周知,针对尖锐几何形状的布接触模拟是应力接触模拟,包括钩住和落下,然后将布板拖过由线段形成的针床(图 9)。由于$h = 0.02s$的大时间步长和 1.0608 的应变极限,布料可以安全地静止在针床上而不会抖动。我们以 1m/s 的速度拉过段尖,观察沿着针床顶部快速滑动,然后离开针床顶部,没有钩住或拉伸伪影(整个应变限制都满足)。同样,如果这些接触以大位移(即大时间步长和/或大速度)在粗糙度上滑动,这些挑战只会增加。在这里,我们测试了C-IPC对极端情况的处理,在这样的情况下,如果我们从尖锐接触中具有极端拉伸,C-IPC的自适应应变限制刚度对于确保时间步长求解在数值上保持可行和高效尤为重要。

\textit{桌布把戏}。经典的桌布技巧试图从桌子设置下拉出一块布,目的是保持餐具直立,理想情况下大部分都保持在原位。使这个技巧起作用的关键是以足够的速度拉动,以便布料的滑动加速度可以克服摩擦力,而设置5上的拉力很小或没有拉力。因此,模拟这种效应需要将弹性和应变限制与摩擦和接触的精确分辨率紧密耦合,以及对锋利边缘滑动接触的可靠处理。这是正确模拟摩擦力、拉力和拉力以及重型餐具法向力之间的平衡所必需的。在图 1 中,我们为 C-IPC 设置了一个表格,其中包含重型和刚性餐具(FEM)。我们模拟拉动应变限制(1.0608 上限)桌布,时间步长为 $h = 0.01s$,速度增加。正如我们预期的那样,当我们从较低的拉动速度开始时,物体不会跟随布料。随着我们提高速度,我们得到的结果就会减少,最后,以 4m/s 的速度拉动,只需稍微移动一下,整个设置保持直立并放在桌子上(我们还注意到,高速拉动的应变限制布开始形成详细的折叠)。最后,通过 8m/s 的高速拉动,桌布平稳拉出,几乎完全不改变设置。在这里,对于后两次成功的拉动,我们观察到来自高速拉动布料的移动边界条件的极端力以及与较重餐具接触时施加的巨大摩擦力。

\textit{扭曲的圆柱体}。最后,我们测试了极端和增加应力下的厚度建模、接触分辨率和屈曲。我们开始用一个1m宽的布筒(0.25m半径),由具有88K节点的壳建模,厚度偏移为1.5mm,d为1mm,时间在$h = 0.04s$。然后,我们同时扭转圆柱体,并分别以 $72^{\circ}/s$ 和 5mm/s 的速度将两侧缓慢移动,以引入起皱、折叠和最终的紧屈曲。为了让这种脚本化的折磨持续整个38秒的序列,我们不应用应变限制 - 因此C-IPC必须在这里解决极高的拉伸力。同样,为了澄清屈曲行为,我们不施加重力(避免下垂)。在模拟的第一秒,我们立即获得有趣的全局折叠(请参阅我们的补充文档和视频)。不久之后,一个厚厚的中央圆筒的缠绕布形成。该圆柱体结构的体积由 C-IPC 的有限厚度偏移支持(图 2 左上角)。为了进行对比,请考虑图2右上角在同一时间步捕获的帧,但现在在没有C-IPC偏移的情况下进行模拟。在这里,我们清楚地看到,如果没有C-IPC的厚度偏移(它也不能捕获材料后来的屈曲行为 - 见下文),就无法形成正确的几何形状,这进一步阐明了一致厚度建模对同维模型的重要性。接下来,随着我们继续我们的 C-IPC 仿真,偏移 32.96 秒的进一步扭曲,我们的偏置加厚布继续支持复杂的行为,包括图 2 底部的最终屈曲几何形状。
