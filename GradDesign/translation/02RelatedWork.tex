\section{相关工作}

\subsection{壳、绳的模拟}

从Terzopoulos等人的开创性工作开始,共维模型的模拟,特别是壳和棒,一直是计算机图形学的重点。有效地模拟布料的复杂行为和头发现在是众多应用中的特别关键的任务。

用于模拟的计算流水线通常采用隐式或半隐式时间积分方法,应用了各种碰撞过滤器和惩罚来帮助解决接触处理问题。

为了提高性能,对 GPU 的支持和多分辨率网格法都在积极探索中。同时,为了提高模型的真度,数据驱动的材料估计,交替空间离散化,甚至是结合纱线和壳模拟的混合模型正被研究中。同样,布料模拟的可微性现在也成为神经网络训练相关应用的关键。

因此,一个关键的挑战是可靠地模拟壳并保证结果。Harmon等人引入了一种用于模拟壳的显式时间步长方法,该方法保证了所有时间步长的不相交。然后,这种方法需要较小的时间步长和简化的摩擦模型。在这里,我们提供了一种互补的、完全隐式的、可微的仿真方法,适用于所有共维模型,具有精确的摩擦接触,保证非相交,以及非反转和(课选的)应变极限满足,可以在所有合理的时间步长下进行。

\subsection{应变限制}

应变限制方法试图对膜变形施加限制。长期以来,人们一直提出了广泛的模型和算法来提供应变限制。为了理解最近方法的行为和局限性,我们按以下方式对它们进行分类:1)约束选择(等式约束或不等式约束);2)受约束的DOF类型(这些通常是基于边缘或奇异值);3)算子拆分模型;4) 支持施加强制约束的求解器。

通过这种细分,我们看到对基于长度的措施的平等约束仍然是一个一致的约束选择。Goldenthal等人对四边形网格边缘长度应用双边约束,并提出了一种快速投影方法来校正预测的位移。English和Bridson等人采用不一致性策略,对三角形边缘中点距离应用相等约束,并扩展快速投影以支持 BDF-2 积分器。Thomaszewski 等人在三角形上应用滞后、旋转同步的小应变来定义等式约束,并通过G-S或Jacobi迭代的后投影法来强制施加应变限制。Chen 和 Tang 同样定义了三角形边长度的等式约束。

通常,可以通过将变形梯度(例如每个三角形的形变梯度)的奇异值约束到固定且有限的范围内,来获得对基于边缘的约束的改进。然而,无论细节如何,相等约束始终保持活动状态,因此简单的自由度计数通常解释了在实施这些应变约束时经常遇到的膜锁定伪影。在这里,替代约束自由度选择可以通过降低约束与自由度的比率来帮助缓解这个问题,但也可以引入新的挑战,例如通过不合格的网格。作为双边的不平等约束的替代方案,以上限和下限的形式早已得到应用,而最近,该约束也被转化为仅考虑上限的情况。但是无论边界如何设置,单边约束只有在其应变措施达到极限时才会激活,因此过度约束系统的可能性降低。

虽然方法明显不同,但我们观察到,除了Chen和Tang的无摩擦弹性静力学的最小二乘近似之外,所有方法目前都采用时间步长分割。在这里,我们看到,在应用约束投影以同时求解接触约束和应变限制之前,通常应用两步法首先求解弹性动力学。或者,引入更多误差,我们还看到三步分裂,其中弹性动力学,接触和应变极限的顺序求解都是解耦的,因此每个时间步独立解决[Narain等人,2012]。因此,对于后续的三步法,其不能同时满足应变极限和接触约束,而对于前两步法,通过分开求解弹性和约束引入的误差也会产生不可接受的伪影;请参阅第 6 节。

最后,我们注意到,据我们所知,没有任何方法(无论约束类型或分裂选择如何)使用约束求解器来保证应变极限的执行。由于约束执行错误在模拟场景甚至单个时间步长中都有所不同,这会导致每个场景和步骤的材料行为发生不可控和不一致的变化。为了实现一致、有效的应变限制C-IPC引入了一个完全耦合的、基于不等式的模型,并严格保证应变极限满足。

\subsection{厚度建模}

通过同维几何建模的薄体的厚度通常比其他尺寸小几个数量级。因此,直接忽略接触厚度是很有吸引力的(也是常见的)。然而,为了正确捕获薄材料的相互作用,我们必须考虑接触厚度的几何效应。我们到处都能看到这种情况,例如当扑克牌堆成一堆(图8)或面条装满碗时(图10)。

捕获厚度的直接策略是硬壳法,该方法使用全平移DOF对薄材料进行体积建模。然而,在这样做的过程中,线性有限元会受到剪切锁定伪影的影响。虽然高阶有限元可以缓解这些收敛问题,但剪切锁定伪影仍然不容易完全避免,而计算成本显着增加。为了应对这些挑战,该方法通常采用假设的自然应变方法的减少集成,以减轻线性单元的剪切锁定。但是,这种增加的复杂性通常不会与共维建模竞争。

为处理共维模型的接触,接触处理策略通常会引入类似厚度的参数来偏移约束,从而减少碰撞处理的不准确性;例如,参见Narain等人和Li 等人以获取最近的例子。然而,正如Li等人所分析的那样,这些厚度无法统一地进行强制施加,而且必须根据场景和数据(例如基于碰撞速度)进行启发式更改,以避免交叉点和不稳定性导致的模拟失败。反过来,虽然它们有时有助于提高接触分辨率,但这些调整后的参数不能可靠地应用于模型一致和不断变化的厚度行为。请参阅第 6.5 节。

我们扩展了IPC模型,以捕获具有偏移屏障的同维域的几何厚度,从而保证与中表面以及中线和点几何形状的最小间距。反过来,这使得在薄接触材料之间的相互作用中对厚度行为进行可靠和一致的建模。

\subsection{连续碰撞检测}

连续碰撞检测(CCD)方法长期以来一直被用作检查相交网格边界的方法。Provot 首先通过三次方程的解来测试共面性,然后执行重叠检查以检测碰撞,从而首次发现线性位移点三角形和边边对的冲击。

到目前为止,浮点CCD的标准公式已经广泛应用并随着技术进步进行了微调。然而,虽然使用浮点运算进行寻根可以提高效率,但重大的数值误差仍然会产生不可接受的错误结果。当距离较小或维数退化时。在这里,虽然从浮点切换到有理数有助于避免舍入问题,但如果不特别小心,成本可能会增加得令人无法接受。

最近的方法解决了CCD的准确性通过应用精确算法来提高鲁棒性与效率。或者,其他人使用要求的保守间隔距离进行保守CCD,以更好地避免相互渗透。后一种策略用于体积元的IPC,使用稳定的浮点CCD实现作为基础求解器。

在这里,为了在C-IPC中建模厚度,我们需要CCD求解,该求解可以保持表面、中线和点元素之间的有限间隔距离。反过来,我们依靠精确地保留偏移表面之间数量级的较小距离来计算精确的接触力。这要求CCD算法有更高的的准确性和稳健性。我们看到所有可用的现有CCD方法和代码中都出现了不可接受的错误,导致故障。我们在标准浮点寻根CCD方法以及最近的精确CCD方法中都看到了这一点。反过来,我们看到这些故障会产生交叉点,减慢收敛速度,并且经常完全停止仿真进度。请参阅第 6.6 节。

作为寻根的替代方案,保守前进(CA)方法已被提出。该算法迭代推进刚体和/或铰接体,直到它们比预定义的小距离更近。从 Mirtich 关于凸刚体的工作 [1996] 开始,这是通过反复计算冲击时间下限(TOI)然后对边界采取保守步骤来实现的。

为了对可变形体轨迹进行鲁棒CCD评估,我们得出了具有任意位移的变形网格基元对的冲击时间下限,并将其应用于CA框架中,以开发一种新的,易于计算的,数值鲁棒的浮点加性CCD(ACCD)算法。在 CA 框架下,ACCD 单调地接近影响时间,没有容易出错的直接寻根。在第6节中,我们确认ACCD在所有其他方法都失败的各种具有挑战性的案例中高效准确地成功,并在浮点CCD方法可以成功的情况下找到类似且通常更好的性能。最后,我们验证了ACCD也适用于C-IPC框架之外的CCD模块,具有更高的效率和鲁棒性。

\subsection{共维建模}

在通用框架中统一仿真所有共维物体,对于仿真效率和准确性至关重要。Martin 等人 专注于统一的弹性模型。它们推导出Elaston,一种高阶积分规则,用于测量沿所有轴的拉伸、剪切、弯曲和扭曲,而不区分余维。Elaston可以精确捕获各种弹塑性行为,而接触力则由逐点惩罚确定,物体由一组球体表示。Chang 等人通过相等约束定义不同共维域之间的所有连接,从而解决混合维弹性体的统一问题。在这里,Bridson等人的碰撞处理算法随后应用于解决接触。物质点法(MPM)还提供了所有协域及其之间接触的通用建模。MPM 离散拉格朗日粒子上的弹性,同时求解欧拉网格自由度上的动量平衡。然后,同维物体之间的接触通过欧拉网格直接解析为速度流。但是,接触中的粘连和间隙误差是MPM中众所周知的伪影,如果网格分辨率太低,则可能是不可接受的。Han等人用拉格朗日自由度对这些伪影进行了研究和缓解。

基于位置的动力学 (PBD)还可以实现具有不同同尺寸的实体的无缝、统一耦合。在这里,本构模型和接触都被解析为约束,随着约束复杂性的增加,这些约束被迭代处理以实现有效的时间积分,但代价是其无法对于精度有准确控制。PBD与XPBD的扩展,即投影动力学,以及通过ADMM的进一步推广都同样提供了协同模拟多种协域的平台。最近的增强功能现在可以改善摩擦力并提高效率。然而,这些方法都缺乏对于迭代收敛的保证,并且采用固定的迭代上限,因此精度和鲁棒性(弹性和接触的分辨率)对计算效率的基本权衡仍然存在。在实践中,这意味着可以而且将会遇到数值不稳定和数值爆炸,特别是在具有挑战性的场景中,例如具有较大的时间步长、刚度、变形或速度,如 Li 等人所示。相反,C-IPC以收敛性和稳定性为目标,能够同时直接模拟所有共维。耦合是通过精确、无相交的接触在任何和所有共维配对之间的相互作用提供的。




