\section{本构应变限制}

在这里,我们首先构建一个新的本构障碍模型,该模型直接执行应变限制,同时保持静止状态的一致性。我们从一种配方开始,该配方增加了现有的膜能量,并在一般各向同性的情况下增加了应变限制电位(第4.1节)。然后,我们在 4.2 节中演示了它的扩展,以直接增强具有应变极限的正交各向异性 STVK 膜。

\subsection{各向同性本构应变限制}

我们将每个元素 $t$ 的各向同性应变限制约束定义为:
\begin{equation}
  \sigma _i ^t < s, \quad \forall i
\end{equation}
每个三角形 t 的变形梯度的奇异值分解为$F^t = U^t\Sigma^t{V^t}^T$,其中$\Sigma^t = \mathrm{diag} (\sigma_1^t, \sigma_2^t)$。约束界$s$表示指定的应变上限,其实际通常选择用于具有 $s\in [1.01, 1.1]$ 的布料。

应变限制需要局部支集。它只应在拉伸过程中当应变接近施加的上限时施加约束力。否则,它应该保持底层膜模型不变。这类似于只有在几乎要接触的情况下才施加的接触力。因此,我们从IPC的接触势障碍开始,构建一个类似的$C^2$平滑、且紧的障碍函数,用于应变限制。对于每个障碍$\sigma_i^t$定义为:
\begin{equation}
  b_i^{st}(x) = \begin{cases}
    -(\frac{\hat s - \sigma_i^t}{s - \hat s})^2\ln(\frac{\hat s - \sigma_i^t}{s - \hat s})& \sigma_i^t > \hat s\\
    0 & \sigma_i^t \le \hat s
  \end{cases}
\end{equation}
因此,仅当应变超过施加的一个小的应变阈值 $s$ 时才会激活。例如,参见势垒能量的插图,应变极限 $s = 1.1$ 和阈值 $\hat s = 1$。

现在,有了障碍函数,我们接下来考虑将它们直接视为约束(如原始障碍和内部点方法)。但是,这样做会简单地将这些障碍相加到元素上,因此在我们更改网格时会获得不一致的行为。相反,为了提供一致的行为,我们以能量密度的形式施加了应变限制,以积分在布料体积上以获得势能函数:
\begin{equation}
  \Psi _{\mathrm{SL}} (x) = \sum_i \kappa_s \int _ \Omega b_i^{st} (x) dV
  \approx \kappa_s \sum_{t, i} V^t b_i^{st}(x)
\end{equation}
其中$\kappa_s$是障碍函数的刚度。对于我们的近似,我们使用$V^t = A^t \xi^t$,其中$A^t$和$\xi^t$分别是三角形t的面积和厚度。我们的最终各向同性的应变限制势能$\Psi_{\mathrm {SL}}$在局部支持下为$C^2$,可以简单地添加到公式\ref{eq3}中的总电位中。反过来,通过优化公式\ref{eq2}中的IPC,在每个时间步直接处理应变限制,同时确保在每个行搜索步骤中不违反应变限制约束。在大多数三角形保持在应变极限阈值以下的步骤中,几乎不需要额外的计算。同时,当拉伸增加时,来自新激活的势垒的必要非零项被添加到系统中,因此,正如我们在第 6 节中所示,在平衡所有施加力的同时严格执行应变限制。

\textit{应变极限刚度}:定义屏障电位后,我们现在指定设置其刚度。对于我们的应变极限模型,我们应用的软化夹紧屏障,其灵感来自IPC对接触势垒能量的平滑。然而,在这里,请注意应变限制的额外成分是应用的 $1/(s − \hat s)^2$ 因子。这种缩放使我们能够将我们的屏障直接应用于极限归一化应变,计算的公式为 $y = (\hat s − \sigma_i^t )/(s −\hat s)$
\begin{equation}
  \hat b_i ^{st}(y) = \begin{cases}
    - y ^ 2\ln(1 + y) & y < 0,\\
    0 & y\ge 0.
  \end{cases}
\end{equation}
反过来,由于我们施加的应变极限($s$)和/或截断阈值($\hat s$)随应用而变化,因此势垒函数,即$y$保持不变。只有从 $\sigma_i^t$ 到 $y$ 的线性映射的梯度变化。这使我们能够在所有不同的应变极限和夹紧阈值的选择上应用单一、一致的初始势垒刚度$\kappa_s$(我们为所有示例使用 $1KPa$)。在这里,障碍势能函数曲线每次都简单地线性重新调整到不同的应变范围,从而为系统提供一致的调节。最后,为了避免当前的应变和需要施加的应变极限之间的微小间隙带来的数值问题(例如,当施加极端边界条件如图9和图1所示时),我们在需要时按照IPC的刚度调整策略调整势垒刚度。从我们的初始$κ^s$开始,每当三角形$t$的应变间隙$s − \sigma_i^t$在两次连续迭代中小于$10^{-4}(s − \hat s)$时,我们将其增加$2\times$(最大界限的$\kappa_s = 0.1MPa$)。请注意,应用此调整时会改变应变限制能量。但是,在极端情况下,这种情况仅发生在实际限制附近。在这里,应变极限力施加约束,而这种适应性提供了改进的数值条件。一旦离开极限,屏障就会恢复到一致的能量,在阈值$s$以下没有任何影响。

\subsection{各向异性本构应变限制}

上面我们已经构建了应变限制作为各向同性本构模型。我们形成了一个可以积分的阻挡能,因此直接添加到势能中,以增强具有硬应变限制的现有膜模型。完成这项工作的关键是应用我们的 $C^2$ 连续函数,以便应变限制的应用不会改变稳定状态下的梯度或Hessian。

或者,我们可以应用类似的本构应变限制策略直接修改膜弹性模型以包括应变极限。为此,我们只需将各向异性膜模型替换为可防止违反应变极限的势垒能,同时匹配原始膜能量梯度和静止时刻的Hessian。

第二种策略的动机是需要将应变限制纳入可用的数据驱动模型中,这些模型已被构建为仔细拟合测量的布料数据。在这里,我们展示了克莱德等人的各向异性且是数据驱动模型的具体应用。其本构模型的能量函数为
\begin{equation}
  \tilde\psi(\tilde E_{11},\tilde E_{12},\tilde E_{22}) = \frac{a_{11}}{2} \eta_1(\tilde E_{11}^2)+\frac{a_{22}}{2}\eta_3(\tilde E_{22}^2)
  + a_{12}\eta_2(\tilde E_{11}\tilde E_{22})+G_{12}\eta_4(\tilde E_{12}^2)
\end{equation}
其中 $\hat E = D^TED$ 是简化并对齐的格林-拉格朗日应变,$D$的列向量是壳在材料空间中的切基和法向基。函数$η$是多项式函数的总和,其实值指数$α _{j i}$ 满足$\eta(0) = 0$ 和 $\eta'(0) = 1$。在这里,第一个约束强制执行零能量、零应力稳定配置,而第二个约束允许参数 $a_{\alpha\beta}$ 和 $G _{12}$ 与无穷小应变下的线性弹性之间的自然对应关系。克莱德等人使用
\begin{equation}
  \eta_j(x) = \sum_{i = 1}^{d_j} \frac{\mu_{ji}}{\alpha_{ji}}((x+1)^{\alpha_{ji}} - 1).
\end{equation}
然后,他们的测量数据仅限于布料断裂极限或弹性范围内的变形。超出此范围,则不太可能进行有意义的外推,因为可能在范围内过度拟合(拟合的指数$α_{ji}$ 的范围可达 $10^5$ )。为了解决这一限制,Clyde等人提出了一种二次外推法进行仿真。然而,这种推断在物理上是没有意义的。为了有效地应用数据驱动的建模,要么应该捕获超出此范围的断裂,要么施加稳定和可控的应变极限以保持在边界内。在这里,我们专注于在尊重基础模型拟合的同时,以可控方式施加应变极限。

\textit{障碍函数定义}。从相同的本构模型开始,我们用障碍函数重新定义$\eta$为
\begin{equation}
  \eta(E) = -E^{\max} \log ((E^{\max} - E) / E^{\max}).
\end{equation}
为了一致性。请注意,此函数需要再次满足$\eta(0) = 0$ 和 $\eta' (0)= 1$,因此对弹性有效。该势垒在 $E^{\max}$ 时,$\eta$也会发散,从而确保 $E$ 永远不会超过测量的应变极限界限。因此,它可以捕获数据驱动的应变极限,同时避免外推。在这里,由于基础模型是各向异性的,因此在各个剪切方向都可以施加不同的测量 $E$ 最大边界,从而能够保留所有应变极限的测量各向异性。

然后,自由模型参数$a_{\alpha\beta}$和$G_{12}$在$\nabla^2\psi(0,0,0)$处与基础模型的拟合匹配。有关详细信息,请参阅补充文档。请注意,这个公式使我们不必计算具有大实指数的敏感(且可能昂贵的)多项式。有关所提出的模型的实验,请参见第 6.2 节。最后,我们注意到,与我们在上一节的各向同性案例中演示的策略(我们直接研究了变形梯度的上限)相反,这里的第二能量对于拉伸和压缩是对称的。

\subsection{用于应变限制的线搜索方法}

如上所述,基于CCD的限制器对于确保线路搜索与我们的接触屏障一致至关重要。现在我们添加了应变限制屏障,我们必须进一步确保每个线搜索也不会违反施加的应变限制。对于各向同性极限,尽管 $2 \times 2$ 个 SVD 具有闭式解,但它不能为可行的步长计算提供多项式方程。因此,在C-IPC线搜索中,在我们计算出无交集起始步长后,如果我们在回溯过程中检测到应变限制违规,我们只需将步长减半。我们重复,直到获得满足能量降低和应变极限的步长。在实践中,因为我们的牛顿步长包括来自Hessian和梯度的应变限制的高阶信息,我们观察到IPC搜索方向在自然避免应变限制方面是有效的。在所有情况下,如果需要,到目前为止,我们观察到最多需要三个回溯步骤就可以限制应变。对于我们的各向异性模型,我们目前也应用相同的回溯策略。然而,虽然目前不需要考虑效率,但我们确实注意到,在这里我们可以制定二次方程,适合直接计算满足步长的最大可行应变极限。














